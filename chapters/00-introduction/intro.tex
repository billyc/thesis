\hypertarget{introduction-main}{%
\section{Introduction: Web-Based Data Visualization for Large-Scale Transport Simulations}
\label{introduction-main}}

I was not a very good student when I started at University. My public high school had prepared me much better than I realized, and my first year of undergraduate engineering study, filled with basic science classes, was mostly review. I mistook this ease for exceptional personal talent; talent which turned out to be fairly average at my well-heeled Ivy League university. I had terrible study habits, was doing poorly and getting bad grades, and by the end of my second year the university suggested I ``take some time off'' to re-evaluate whether I was really interested in an engineering degree.

It was through this lens that I found a summer internship in my hometown outside of Washington, D.C.. It was a small firm specializing in spending U.S. government grants from the Departments of Transportation and Commerce, building transportation models for various cities across the country. This was a revelation: I could combine my already-existent love of computers and technology with poring over city maps of bus and rail lines, and then converting those maps into networks of nodes and links (edges) that were the transportation supply inputs to travel demand models. Back then, the models were ``aggregate four step models,'' a term that sounded extremely advanced but in hindsight left a lot out of the equations. Nonetheless, I found my calling and I have been building, running, and interpreting transportation model simulations ever since. I also got that engineering degree with ease, once I knew what I wanted to do with it.

Early in that progression, it was clear that decisionmakers had little interest in looking at tables of numbers and often didn't even have the time to interpret a multi-variate plot with anything other than ridership on the Y-axis and year on the X. They wanted maps. There was a pivotal late night in a hotel room in Pittsburgh, Pennsylvania, where my boss and I spent hours coloring in paper maps with colored pencils, to show the model's predictions for which neighborhoods would benefit (or not) from a proposed busway investment. ``Never again,'' my boss exclaimed, and the very first day after that meeting, we bought some inexpensive (but proprietary) thematic mapping software that ran on ancient Windows 3.0. Those were the heady days before Geographic Information Systems (GIS) were widely available, but even then the old adage \emph{A picture is worth a thousand words} was wise indeed.

The travel forecasting profession is now well established worldwide, and transport simulation software is almost always married to some sort of visualization package to support "informed decisionmaking" -- meaning, the simulation outputs are too vast, detailed and complex to be interpreted directly, and must instead be distilled into summaries, maps, and "stories" that explain what the model is trying to tell you. This informed decisionmaking is the ultimate goal of most transport studies: we do not build these simulations just for the fun of it (although it is fun building them) -- we always have some sort of policy question in mind, and we hope that the computer simulations can help us choose wisely.

This leads to a quandary. Most often, transport investments and policies are government-initiated, and involve spending massive amounts of taxpayer dollars. How does an informed citizenry interface with these complex tools? If the simulation software or the visualization packages are based on proprietary software, then it creates a second-order "ivory tower" problem where only the experts can code, review, and summarize simulation outputs. This causes obvious conflicts with regard to transparency and government accountability. Many government agencies therefore demand or at least encourage that their tools be open source, meaning the inputs, the code, and the outputs be accessible to outsiders.

And now we come to the topic of this thesis. It has been my lifelong career goal to make transport simulation software open, accessible, and understandable to the greater public as well as to its usual government consumers. I have been continually developing various user interfaces to transport modeling software since the early 2000s. With the recent visibility of the \gls{MATSim} framework, there is an excellent simulation platform on which to build. But MATSim has only very limited open source visualization tools.

What can be done to provide a modern visualization platform to complement MATSim and other large-scale simulations?

The journey toward finding the answer to that question is the story documented in this dissertation. In recent years, the graphical and processing capabilities of standard desktop web browser software have become so advanced and ubiquitous, it seems prudent to explore whether a web browser could be that platform. How far can one go with the modern web platform as the basis for opening up the world of transportation microsimulation to the larger world: to anyone on the Internet with a web browser?

\textbf{The contribution of this dissertation to the scientific body of knowledge is straightforward: first, it documents the evaluation of web-browser based data visualization techniques that integrate with open-source transport microsimulation tools such as MATSim and ActivitySim, and then it describes a new, complete, and unique open-source web platform that allows transport researchers to produce data visualizations and dashboards to support informed decisionmaking.}

The dissertation is organized into three parts. Part I describes the current state of research and initial experiments using standard desktop web browsers to display MATSim transport simulation inputs and outputs. Several small tools explore the datasets associated with MATSim simulations such as transit networks and accessibility data. Then, the knowledge gained from these experiments is synthesized into larger client/server web applications, with "front-end" code running in the local web browser and "back-end" server code that handles file storage, post-processing of data, and configuration details. For reasons described in \ref{ch:mathub}, this approach has some disadvantages which hindered adoption, particularly with respect to our ability to rapidly iterate the platform features.

Part II describes a rethinking of the platform design which essentially eliminates the server-side capabilities and moves almost everything into a client-side web application. The grim realities of 2020 required our research team to apply the microsimulation models to a new field (urban virus propagation), and a visualization tool was needed immediately and desperately to compare thousands of scenarios. The best parts of what is described in Part I were reformulated into a much more nimble tool which did not require any back end server capabilities beyond simple file storage. This approach was so successful that, once there was bandwidth on our team to turn back to transport studies, it was applied to more typical transport studies invoving demand-responsive transport scenarious. All of these individual project applications are documented in Part II.

Finally, Part III describes the synthesis of the above project-based portals into a cohesive and general web-based platform for building interactive dashboards for any sort of transportation microsimulation needs. \ref{ch:simwrapper} describes this platform, which is called "SimWrapper". The design and capabilities of SimWrapper are fully explored, along with considerations based on direct user feedback from users of the platform. Then \ref{ch:simwrapper-sites} exposits further with some real-world applications of the SimWrapper platform. The dissertation concludes with user feedback based on interviews with users of SimWrapper, and some follow-on discussion of next steps, possible research avenues for continuation of this line of research in the future, and discussion of some larger questions about whether there is much "market space" for an open source tool such as SimWrapper, given the wide array of commercial tools already available.

My personal desire is to activate and motivate a community of transport planners and agency staff who wish to advance the state of the practice with open source tools. The civic process withers in darkness, and open tools can work hand in hand with open data to dispel cynicism about government. Not to mention that a new generation of transport planners want to get their hands on data and tools that they can crack open, dissect, and modify in ways that we can only imagine. Only open source tools can satisfy that curiosity.

Let's begin.

