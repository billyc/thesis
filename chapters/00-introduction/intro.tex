\hypertarget{introduction-main}{%
\section{Web-Based Data Visualization in Support of Agent-Based Microsimulation Models}
\label{introduction-main}}

Complex, large-scale transportation simulations have historically resided in a specialized domain where only the experts can code, review, and summarize simulation outputs.  But what if such complex systems and volumes of data could be accessed by a complementary, modern visualization platform, ``translating'' vast stores of indigestible information into open, accessible, and understandable tools for the public and, importantly, policymakers tasked with decision-making and spending on society's transportation needs?  The journey toward finding the answer to that question is documented in this dissertation.

On a personal note, before getting to the research itself, I want to remark on how this research came to pass. First, I was a rather unexceptional student when I started at University. My public high school had prepared me much better than I realized, and my first year of undergraduate engineering study, filled with basic science classes, was mostly review. I mistook this ease for \emph{exceptional personal talent}; talent which turned out to be fine but average at this well-heeled Ivy League university. I developed terrible study habits which led to bad grades when new topics came around. By the end of my second year, the university suggested I should ``take some time off'' to reevaluate whether I was really interested in an engineering degree.

It was through this lens that I found a summer internship in my hometown outside of Washington, D.C.. It was a small firm specializing in spending U.S. government grants from the Departments of Transportation and Commerce, building transportation forecasting models for various cities across the country. This was a revelation: I could combine my already-extant love of computers and technology with poring over city maps of bus and rail lines, and then convert those maps into networks of nodes and links (edges) that were the transportation supply inputs to the models. Back then, the models were ``aggregate four step models,'' a term that sounded extremely advanced but in hindsight left a lot out of the equations. Nonetheless, I found my calling and I have been building, running, and interpreting transportation model simulations ever since. I also got that engineering degree with ease, once I knew what I wanted to do with it.

Early in that progression, it became obvious that decisionmakers had little interest in looking at tables of numbers and often did not even have the time to interpret a multi-variate plot with anything other than ridership on the Y-axis and year on the X. They wanted maps. There was a pivotal late night in a hotel room in Pittsburgh, Pennsylvania, where my boss and I spent hours coloring in paper maps with colored pencils, to show the model's predictions for which neighborhoods would benefit (or not) from a proposed busway investment. ``Never again,'' my boss exclaimed, and the very first day after that meeting, we purchased some inexpensive thematic mapping software that ran on ancient Windows 3.0. Those were the heady days before Geographic Information Systems (GIS) were widely available, but even then the old adage \emph{A picture is worth a thousand words} was wise indeed.

The travel forecasting profession is now well established worldwide, and transport simulation software is almost always married to some sort of visualization package to support ``informed decisionmaking'' -- meaning, the simulation outputs are too vast, detailed and complex to be interpreted directly, and must instead be distilled into summaries, maps, and stories that explain what the model is trying to tell you. This informed decisionmaking is the ultimate goal of most transport studies: we do not build these simulations just for the fun of it (although it is fun building them) -- we always have some sort of policy question in mind, and we hope that the computer simulations can help us choose wisely.

This leads to a quandary. Most often, transport investments and policies are government-initiated, and involve spending massive amounts of taxpayer or investor money. How does an informed citizenry interface with these complex tools? If the simulation software or the visualization packages are based on proprietary software, it creates a second order ``ivory tower'' problem where only the experts can code, review, and summarize simulation outputs. This causes obvious conflicts with regard to transparency and government accountability. Many government agencies therefore demand or at least encourage that their tools be fully open: meaning the inputs, the code, and the outputs must be accessible to outsiders.

And now we come to the topic of this thesis. It has been my lifelong career goal to make transport simulation software open, accessible, and understandable to the greater public as well as to its usual government consumers. I have been continually developing various user interfaces to transport modeling software since the early 2000s. With the recent visibility of the \gls{MATSim} microsimulation framework, there is an excellent simulation platform on which to build. But MATSim has only very limited open source visualization tools.

What can be done to provide a modern visualization platform to complement MATSim and other large-scale simulations?

One possible solution is presented in this dissertation. In recent years, the graphical and processing capabilities of standard desktop web browser software have become so advanced and ubiquitous, it seems prudent to explore whether a web browser could be that platform. How far can one go with the modern web platform as the basis for opening up the world of transportation microsimulation to the larger world -- to anyone on the Internet with a web browser?

\textbf{The contribution of this dissertation to the scientific body of knowledge is straightforward: first, it documents the development and evaluation of web-browser based data visualization techniques that integrate with open-source transport microsimulation tools such as MATSim and ActivitySim. Then it describes a new, complete, and unique open-source web platform that allows transport researchers to produce data visualizations and dashboards to support informed decisionmaking.}

The dissertation is organized into three parts. Part \ref{part:1servers} describes the current state of research and initial experiments using standard desktop web browsers to display MATSim transport simulation inputs and outputs. Chapter \ref{ch:server-experiments} details several small tools that explore the datasets associated with MATSim simulations such as transit networks and accessibility data. Then, the knowledge gained from these experiments is synthesized into larger client/server web applications described in chapter \ref{ch:mathub}, with ``front-end'' code running in the local web browser and ``back-end'' server code that handles file storage, post-processing of data, and configuration details. For reasons described in that chapter, this approach had some disadvantages which hindered adoption, particularly with respect to our ability to rapidly iterate the platform's features.

Part II describes a rethinking of the platform design which essentially eliminates the server-side capabilities and moves almost everything into a web browser application. The grim global realities of 2020 required our research team to apply the microsimulation models to a new field (urban virus propagation), for which a visualization tool was needed immediately and desperately to compare thousands of COVID-19 scenarios, the subject of chapter \ref{ch:covid-sim}. The lessons learned in Part I were timely, and pieces of that early research were reconstituted into a much more nimble tool -- one that did not require any back-end server capabilities beyond simple file storage. This approach was so successful that, once there was bandwidth to return our attention to transport studies, it was applied to transportation topics such as demand-responsive transport in chapters \ref{ch:avov} and \ref{ch:pave}. Thus three individual project applications are documented in Part II.

If you repeat the same manual process three times, it is ripe for refactoring into a more general form. Part III describes the synthesis of the above project portals into a cohesive and general web-based platform for building interactive data dashboards for any sort of transportation microsimulation scenario. Chapter \ref{ch:simwrapper} describes this platform, which is called ``SimWrapper''. The design and capabilities of SimWrapper are fully explored, along with considerations based on direct user feedback from users of the platform. Then chapter \ref{ch:simwrapper-sites} exposits further with some real-world applications of the SimWrapper platform. The dissertation concludes in chapter \ref{ch:conclusion} with user feedback from users of SimWrapper; some follow-on discussion of next steps; possible research avenues for continuation of this line of research in the future; and discussion of some larger questions about whether there is much market space for an open source tool such as SimWrapper, given the presence of commercial tools and larger open source projects that offer some similar capabilities.

Two appendices are also included for further reading.

\autoref{appendix:python-r} describes complementary tools that support the SimWrapper platform; MATSim helper packages in the Python and R languages were built as part of this research. These packages help post-process simulation results to prepare them for ingestion by the visualization platform.

\autoref{appendix:simwrapper} describes many of the individual visualizations available in SimWrapper. This is comprehensive at the time of publication, but as the tool is an ongoing open source project with continuous feature development, this appendix must be considered as a simple snapshot of current capabilities.

My personal desire is to activate and motivate a community of transport planners and public agency staff who wish to advance the state of the practice of transport planning using open source tools. The civic process withers in darkness, and open tools can work hand-in-hand with open data to dispel cynicism about government. Not to mention that a new generation of transport planners want to get their hands on data and tools that they can crack open, dissect, and modify in ways that we can barely imagine. Only open source tools can satisfy that curiosity.
