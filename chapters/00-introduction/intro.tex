\hypertarget{introduction-main}{%
\section{Introduction: Web-Based Data Visualization for Large-Scale Simulations}
\label{introduction-main}}

- motivation
- research questions
- outline



% ----------- extra formatting help

The COVID-Sim website is available at \url{https://covid-sim.info}

\hypertarget{how-websites-are-built-clients-and-servers}{%
\subsection{How websites are built: clients and
servers}\label{how-websites-are-built-clients-and-servers}}

Of course, much more complex arrangements are also possible. The web
server can run code or scripts which generate part of the page
dynamically, can call APIs which pull data from other servers or
databases, and so on. These are known as ``dynamic'' sites, and for
example our MatHub project used this arrangement.

\hypertarget{single-page-applications}{%
\subsection{Single Page Applications}\label{single-page-applications}}

As early as 2003 {[}YY reference SPA page on wikipeda -- better ref?{]},
the concept of a single-page application which runs Javascript code in
the browser was already in existence. These types of sites run
Javascript locally in the browser to transform the page contents that
arrive from the web server. This is often done to make a page feel more
interactive or more like a native desktop app. Some of the most popular
websites in existence, such as Twitter and Facebook, employ this
architecture.

With all these pieces in place, the unique overall architecture of the
EpiSim data visualization portal emerges:

\begin{itemize}
\item
  An ``SPA'' single page application, based on Vuejs and hosted on a
  static website hosting provider
\item
  Hierarchical file storage on a university departmental file server,
  with data for all published runs available via HTTP. Each run is
  stored in its own folder, and contains:

  \begin{itemize}
  \item
    Automatically-generated configuration files, produced when the
    EpiSim simulation runs are set up, which describing the specific
    combinations of input parameters that are to be run.
  \item
    A lookup table which maps each combination of input parameters to a
    specific output dataset
  \item
    Zipped output files containing the summary data for each simulation
    run identifiable in the above lookup table.
  \end{itemize}
\end{itemize}

\ldots{} YY

