% !TeX spellcheck = en_US
% !TEX root = ../phd.tex	

\section{Motivation}
\label{sec:intro-Motivation}
\todo[inline]{Motivation schreiben}

%%%
%%%% 
%%From IK:
%\textcolor{cyan}{The remainder of this \nameCref{Einleitung:ch} is structured as follows. \cref{introduction-sec:pricing} addresses the basic relevant economic concepts of external effects and pricing.
%\cref{sec:research-questions} provides the research questions posed in this thesis. In \cref{sec:research-approach}, the general research approach is presented, including the key features of the agent-based and dynamic pricing approach. An overview of the applied simulation framework is provided in \cref{introduction-sec:matsim}. The outline of this thesis and overview of simulation experiments and applied case studies are given in \cref{sec:outline}.}
%%Ende IK:


\section{Forschungsfragen}
\label{sec:intro-Forschungsfragen}
\todo[inline]{Forschungsfragen}
%%%% 
%%From IK:
%Starting from a more general perspective, the first research question addresses the overall functionality of the proposed pricing methodology:
%Is simulated dynamic pricing a useful approach for the optimization of transport systems?
%%
%In this context, the next research question explores both the effectiveness and effects of the pricing methodology: How do congestion pricing or noise pricing change travel behavior and is there a positive impact on the overall system welfare?
%%
%The following research question directly builds on the previous one and addresses the interrelation of different external effects: How are congestion, noise and air pollution correlated with each other; or, what impact does the pricing of a single external effect have on the others?
%%
%A further research question investigates the design of the simulation-based pricing methodology: What are the implications for the design of a pricing strategy resulting from a simulation framework which allows for real-world case studies and stochastic user behavior?
%%
%Additionally, varying assumptions regarding transport users' travel behavior are tackled: How do simulation results change if transport users' mode choice decisions are incorporated into the transport model?
%%
%Finally, optimal pricing strategies are addressed in the context of innovative modes of transportation: How may simulated dynamic pricing contribute to an improvement of the transport system in regard to the prospect of \glspl{SAV}?

\section{Forschungsansatz}
\label{sec:intro-Forschungsansatz}
\todo[inline]{Forschungsansatz}

\section{Gliederung}
\label{sec:intro-Gliederung}
\todo[inline]{Forschungsansatz}

Diese Arbeit ist wie folgt aufgebaut. \cref{part:Grundlagen} enthält die der weiteren Arbeit zugrunde liegenden Rahmenbedingen und verwendete Software.


%The thesis is structured as follows. \cref{part:congestion} addresses the simulation-based computation of congestion prices in the \gls{pt} mode (\cref{ptDelayPricing:ch}) and car mode (\cref{vtts:ch}, \cref{decongestion:ch}).
%\cref{part:noise} describes the computation of noise damages (\cref{noise:ch}) and noise prices (\cref{internalizationACP:ch}, \cref{internalizationMCP:ch}). In \cref{part:severalExternalEffects}, prices are set accounting for several external effects (congestion and noise in \cref{simultaneous:ch}; congestion, noise and air pollution costs in \cref{cna:ch}). \cref{part:avOpt} addresses optimal pricing strategies for \glspl{SAV}. Finally, \cref{chapter:conclusion} provides an overall conclusion and outlook.
%
