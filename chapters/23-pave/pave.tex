%% ----- INTRODUCTION -----
\section{Introduction}
\label{pave-intro}

PAVE -- we took the winningness of AVOV and embiggened it.

The same combination of curated runs stored on a file server, with a custom front-end designed to be outward facing for public outreach meetings with government agencies and the wider public audience.

%% ----- PROJECT DESCRIPTION -----
\section{Project Description}
\label{pave-project-description}

%% ----- UNIQUE SITE FEATURES -----
\section{Website Implementation}
\label{pave-site-features}

- Interactive animation of DRT vehicles driving around, using color
- Custom scenario picker in the site navigation (left bar), which lets user select a small or large DRT service area as well as pricing and other service levels.
- Upon the user selection, the results panel updates to show the precomputed summary statistics for that specific combination of model parameters
- Carousel of mode shift and pie charts
- Bespoke panel showing Key Performance Indicators (KPI) for the chosen run

%% ----- FEEDBACK -----
\section{Public Outreach Results and Feedback}
\label{pave-feedback}

- Public was not super into it

% YY get quotes from Tilmann

%% ----- DISCUSSION AND OUTLOOK -----
\section{Discussion and Outlook}
\label{pave-discussion}

%% ----- SUMMARY -----
\section{Summary}
\label{pave-summary}

The PAVE project comprised the third project site built on a common foundation of web-based visualization technologies and HTTP-accessible file storage for site configuration and simulation outputs. Several novel components were built for the project, including visual navigation for scenario selection, a consistent key performance indicator output panel, and a carousel for collating related charts in order to reduce visual clutter.

Feedback from users was positive. Internal staff considered the website well-conceived, but the level of effort to produce this bespoke site was high. This led to questions and suggestions that the features might be better integrated in a more general platform instead of a project-specific website. The development of a generic data visualization platform in this vein is the subject of the following chapters in Part III.
