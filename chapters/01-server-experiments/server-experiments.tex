%% ----- INTRODUCTION -----
\hypertarget{server-experiments-introduction}{%
\section{Introduction}\label{server-experiments-intro}}

- Background: before arriving at VSP I worked on internal model workflow projects (PSRC) and data visualization portals (TNCs today, SFCTA)
- Interest in open, web-based platforms for displaying and analyzing transport model outputs. So we started playing.
- Discussed with users what would be useful

%% ----- TRANSIT VIEWER -----
\hypertarget{server-experiments-transit}{%
\section{MATSim Transit Network Viewer}\label{server-experiments-transit}}

The first step in building web-based tools for MATSim was reading and parsing MATSim input files. Specifically, MATSim inputs representing the transportation network and public transit supply are both well-defined XML formats YY and are of a size that does not present problems on typical computers. Based on feedback from team members, a transit service explorer was identified as a good first task.

A minimally-useful tool needed the following capabilities:

\begin{itemize}
  \tightlist
  \item
    Load and parse a MATSim network XML file and transit supply input file, including gzip-compressed files, as gzip is typically used to compress MATSim files
  \item
    Display the network links on a zoomable background map
  \item
    Display the transit lines and routes that are on network links, using width and color to depict multiple routes or transit modes on a facility
  \item
    Allow the user to select a specific facility (link) and see the details of the transit routes and lines which use it
\end{itemize}

This seemingly small project presented some initial challenges. First and foremost, by design and to protect the security of a user's files, by default web browsers do not allow any access to the files on a users local computer.

- Talk about the file server
- Talk about coordinates
- Talk about standalone tools using Electron

%% ----- GEOSERVER -----
\hypertarget{server-experiments-geoserver}{%
\section{Accessibility maps using GeoServer}\label{server-experiments-geoserver}}

- GeoServer: accessibility for Kenya

%% ----- EMISSIONS  -----
\hypertarget{server-experiments-emissions}{%
\section{MATSim Emissions}\label{server-experiments-emissions}}

%% ----- DISCUSSION AND OUTLOOK  -----
\hypertarget{server-experiments-findings}{%
\section{Discussion and Outlook}\label{server-experiments-findings}}

%% ----- SUMMARY  -----
\hypertarget{server-experiments-summary}{%
\section{Summary}\label{server-experiments-summary}}


Internal use of tools was very limited because of lack of features (could do it all in QGis or VIA); and also because the individual tools were haphazard in their installation. Each one had a separate URL, or required files to be uploaded somewhere users were unaccustomed to.

Through many discussions we decided that the experiments had been useful in pointing us toward building a centralized data analysis platform.

- Show progression from PHD seminar slides

- This led to MatHub: a fully centralized, client/server based approach to creating standard data visualizations.

Next chapter describes that platform fully.


