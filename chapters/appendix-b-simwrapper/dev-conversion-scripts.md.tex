\emph{For getting MATSim outputs into SimWrapper}

In general, most MATSim outputs are too big and use unwieldy formats.
The scripts below can help convert some MATSim outputs into smaller,
more internet-friendly formats.

If you develop your own scripts, please let us know and we can add them
here!

\begin{center}\rule{0.5\linewidth}{0.5pt}\end{center}

\hypertarget{create-json-network.py}{%
\subsection{create-json-network.py}\label{create-json-network.py}}

Create a JSON network appropriate for loading into SimWrapper
visualizations

\begin{itemize}
\tightlist
\item
  Download script here:
  \textbf{\href{https://raw.githubusercontent.com/simwrapper/simwrapper/master/scripts/create-geojson-network.py}{create-json-network.py}}
\end{itemize}

\textbf{Command:}

\texttt{python\ create-json-network.py\ {[}network{]}\ {[}coord-system{]}}

\textbf{Inputs:} MATSim network.xml.gz file; coordinate system

\textbf{Outputs:} \texttt{network.json.gz} which loads into SimWrapper
much faster than an \texttt{.xml.gz} file

\begin{center}\rule{0.5\linewidth}{0.5pt}\end{center}

\hypertarget{create-csv-network.py}{%
\subsection{create-csv-network.py}\label{create-csv-network.py}}

\begin{itemize}
\tightlist
\item
  Download script here:
  \textbf{\href{https://raw.githubusercontent.com/simwrapper/simwrapper/master/scripts/create-csv-network.py}{create-csv-network.py}}
\end{itemize}

\emph{Create a CSV network appropriate for loading into R with the
\texttt{sfnetworks} package}

\textbf{Command:}

\texttt{python3\ create-csv-network.py\ {[}my-network.xml.gz{]}}

\textbf{Inputs:} MATSim network.xml.gz file

\textbf{Outputs:} network.csv file with one row per link. Includes a
\texttt{wkt} column with the WKT LINESTRING field, from which an R
sfnetwork can be created.

Sample code to read the output CSV into R:

\begin{Shaded}
\begin{Highlighting}[]
\FunctionTok{library}\NormalTok{(tidyverse)}
\FunctionTok{library}\NormalTok{(sfnetworks)}
\FunctionTok{library}\NormalTok{(sf)}

\NormalTok{sf }\OtherTok{\textless{}{-}} \FunctionTok{st\_as\_sf}\NormalTok{(}\FunctionTok{read\_csv}\NormalTok{(}\StringTok{"my{-}network.csv"}\NormalTok{), }\AttributeTok{wkt=}\StringTok{"wkt"}\NormalTok{, }\AttributeTok{crs=}\DecValTok{25832}\NormalTok{)}
\NormalTok{network }\OtherTok{\textless{}{-}} \FunctionTok{as\_sfnetwork}\NormalTok{(sf)}

\FunctionTok{ggplot}\NormalTok{() }\SpecialCharTok{+}
  \FunctionTok{geom\_sf}\NormalTok{(}\AttributeTok{data=}\FunctionTok{st\_as\_sf}\NormalTok{(filtered, }\StringTok{"edges"}\NormalTok{), }\AttributeTok{col=}\StringTok{"grey50"}\NormalTok{) }\SpecialCharTok{+}
  \FunctionTok{geom\_sf}\NormalTok{(}\AttributeTok{data=}\FunctionTok{st\_as\_sf}\NormalTok{(filtered, }\StringTok{"nodes"}\NormalTok{), }\FunctionTok{aes}\NormalTok{(}\AttributeTok{size=}\DecValTok{1}\NormalTok{)}
\end{Highlighting}
\end{Shaded}

\begin{center}\rule{0.5\linewidth}{0.5pt}\end{center}

\hypertarget{parse-drt-link-events.py}{%
\subsection{parse-drt-link-events.py}\label{parse-drt-link-events.py}}

Parse the event file containing DRT events.

\begin{itemize}
\tightlist
\item
  Download script here:
  \textbf{\href{https://raw.githubusercontent.com/simwrapper/simwrapper/master/scripts/parse-drt-link-events.py}{parse-drt-link-events.py}}
\end{itemize}

\textbf{Command:}
\texttt{python\ parse-drt-link-events.py\ {[}network{]}\ {[}events{]}\ {[}coord-system{]}}

\textbf{Inputs:} network.xml.gz file; events.xml.gz file; a valid
coordinate system

The command will run much faster if you filter the events file to only
contain drt events first. You can use \texttt{grep} or a similar tool,
with a command like:

\begin{itemize}
\tightlist
\item
  \texttt{zcat\ events.xml.gz\ \textbar{}\ grep\ "drt"\ \textgreater{}\ drt-events.xml}
\item
  But that strips the xml header and footer so actually doesn't work.
  You need the \texttt{\textless{}events\textgreater{}} and
  \texttt{\textless{}/events\textgreater{}} tags. I'll fix this soon..?
\end{itemize}

\textbf{Outputs:} \texttt{drt-vehicles.json}

Use gzip to compress that output so things load faster.
