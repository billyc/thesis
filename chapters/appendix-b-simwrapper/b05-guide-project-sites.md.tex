You can hide all of the SimWrapper ``chrome'' such as the folder
browser, and provide custom header/footer for each page using CSS. This
is useful for building special-purpose tools that might be
outward-facing, for example.

\hypertarget{setting-up-a-project-site}{%
\subsection{Setting up a project site}\label{setting-up-a-project-site}}

\begin{itemize}
\tightlist
\item
  Create \texttt{simwrapper-config.yaml} in your project folder
\item
  Define the custom footer and header in markdown files
\item
  Use css in a custom CSS file to present as you wish
\end{itemize}

There are two general configuration options:

\textbf{hideLeftBar:} True/false, hides the left-side folder browser
panel

\textbf{fullWidth:} True/false, true removes the fixed-width centered
panel if you want a full-screen experience

\hypertarget{sample-simwrapper-config.yaml-file}{%
\paragraph{Sample simwrapper-config.yaml
file}\label{sample-simwrapper-config.yaml-file}}

\begin{lstlisting}
  hideLeftBar: true
  fullWidth: true
  header: header.md
  footer_en: footer.md
  footer_de: footer.md
  css: custom.css
\end{lstlisting}

\paragraph{Sample header.md}
Sample header.

\begin{lstlisting}
  !-- header image logo -->
  <img class="project-logo"
       src="https://svn.vsp.tu-berlin.de/repos/public-svn/matsim/scenarios/countries/de/kelheim/projects/KelRide/logos/KelRide-text.png"
  />
\end{lstlisting}

\paragraph{Sample footer.md}
Sample footer.

\begin{lstlisting}
  <footer>
  <div class="container">
    <div class="logos">
        <img src="https://svn.vsp.tu-berlin.de/repos/public-svn/matsim/scenarios/countries/de/kelheim/projects/KelRide/logos/KelRide-text.png"/>
        <img src="https://svn.vsp.tu-berlin.de/repos/public-svn/matsim/scenarios/countries/de/kelheim/projects/KelRide/logos/LK_Kelheim.png"/>
        <img src="https://svn.vsp.tu-berlin.de/repos/public-svn/matsim/scenarios/countries//de/duesseldorf/projects/komodnext/website/logos/TU.svg"/>
    </div>

    <div class="menu">
      VSP / TU Berlin
      (c) 2023 TU Berlin. <a href="https://vsp.berlin/impressum">Impressum</a>
    </div>
  </div>
  </footer>

\end{lstlisting}

\paragraph{Sample custom.css}

Custom CSS can be quite extensive, refer to the online documentation for examples.

