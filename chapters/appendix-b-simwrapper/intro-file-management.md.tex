\textbf{SimWrapper} doesn't have a back-end database; it reads the files
in folders that you give it access to.

Depending on where your files are, this may require some configuration!

\begin{itemize}
\tightlist
\item
  At TU Berlin, everything in our public subversion server is already
  accessible. Check it out here:
  \href{https://vsp.berlin/simwrapper}{vsp.berlin/simwrapper}
\item
  You can view files on your local computer, see below
\item
  If you have SSH access to network compute-cluster servers, you can
  mount the remote filesystems and they behave as if they are local; see
  below
\item
  You can also set up remote file storage on cloud servers such as
  Amazon etc.
\end{itemize}

\hypertarget{local-folders-on-your-computer}{%
\subsection{Local folders on your
computer}\label{local-folders-on-your-computer}}

\textbf{Easy version:} Use Google Chrome or Microsoft Edge to access
files on your local machine. These two browsers have a built-in file
access API which lets you grant access to a folder for viewing, and then
everything just works. From the SimWrapper home page, click
\texttt{Add\ Local\ folder...} to get started.

\textbf{Hard version:} Other browsers (Safari, Firefox, others) block
local file access by default. The only way to access local files is by
running a small local HTTP server on ``localhost:8000''. This works for
all browsers but Safari has additional measures which make it the worst
choice for this.

For these reasons, we strongly recommend using Google Chrome (or Edge if
that's your thing)

We formerly supported both Python and Java versions of this HTTP server,
but now only the Python 3.x version is supported.

\textbf{Python:} this should work with Python 3.6+:

\begin{enumerate}
\def\labelenumi{\arabic{enumi}.}
\tightlist
\item
  Run \texttt{pip3\ install\ simwrapper} to install the simwrapper
  command-line tool. If you don't have \texttt{pip} installed, you'll
  need to get that set up in your Python environment first.
\item
  \texttt{cd} to the folder you wish to serve, and then run the command
\item
  \texttt{simwrapper\ serve}
\item
  Test that it's working by browsing to \url{http://localhost:8000}. If
  you see a file listing, then it is working
\item
  You can now go to the SimWrapper website and choose ``Local files'' to
  browse your folders inside of SimWrapper!
\item
  You can run multiple copies of the SimWrapper Python tool in separate
  root folders if you wish. Each one will run on the next port, so
  localhost:8001, localhost:8002, etc.
\end{enumerate}

\hypertarget{mounting-remote-file-systems-cluster-compute-services-etc}{%
\subsection{Mounting remote file systems -- cluster compute services
etc}\label{mounting-remote-file-systems-cluster-compute-services-etc}}

You can ``virtually'' mount remote filesystems using the tool
\texttt{sshfs}. It creates an ssh tunnel to the remote machine using
your username/login credentials, and mounts the files it finds there
under a subfolder on your machine.

Once the sshfs tunnel is established, you can browse the files there as
if the files are local on your machine, as above.

\begin{itemize}
\tightlist
\item
  Linux: install \texttt{sshfs} and follow your OS instructions for how
  to mount remote filesystems
\item
  Mac: Use \href{https://osxfuse.github.io/}{MacFUSE} to set up sshfs
\item
  Windows: \href{https://winfsp.dev/}{WinFSP} may provide this; please
  let me know if this works for you
\end{itemize}

\hypertarget{internetcloud-storage}{%
\subsection{Internet/cloud storage}\label{internetcloud-storage}}

SimWrapper can be configured to use any Internet storage that can serve
up file and folder listings similar to Apache or NGINX. Currently, this
means public storage only; no authenticated storage. \emph{{[}to be
enhanced in the future!{]}}

\begin{itemize}
\item
  To set up SimWrapper for internet storage, you need to fork SimWrapper
  and set up your own instance, then define your storage endpoint in the
  file \texttt{src/fileSystemConfig.js} following the examples there.
\item
  If you do not yet have your own instance of SimWrapper set up, follow
  the \href{dev-developing-simwrapper.md}{instructions here}.
\end{itemize}

\textbf{Amazon AWS} You can set up access to Amazon EC2/EFS file storage
by following this guide:
\url{https://docs.aws.amazon.com/efs/latest/ug/wt2-apache-web-server.html}

