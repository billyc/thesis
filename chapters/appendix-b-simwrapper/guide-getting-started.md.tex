Welcome to SimWrapper! Let's get you up and running with the basics.

\begin{itemize}
\tightlist
\item
  This guide uses sample data that's hopefully a lot like the data you
  would have for your projects
\item
  Use Google Chrome or MS Edge for the guide. Other browsers (Firefox,
  Safari) require a \href{file-management}{separate local HTTP server}
  to access local files; for now that's just a stumbling block to
  getting started.
\end{itemize}

\hypertarget{how-it-works-simwrapper-and-file-based-configuration}{%
\subsection{How it works: SimWrapper and file-based
configuration}\label{how-it-works-simwrapper-and-file-based-configuration}}

Most MATSim/ActivitySim outputs such as the \texttt{*.xml.gz} files are
too large to open in a web browser, so SimWrapper provides a set of
\emph{visualization plugins} which can display something useful for you.
Plugins exist for lots of things and the list is growing: link volumes,
agent animations, aggregate area summaries, and more.

Here's how it works: For every visualization you want to create, you
write a small \emph{configuration file} and store it in the same folder
as the inputs for that visualization. We use the YAML text format, which
is a common configuration file format. For each properly named YAML
file, one visualization thumbnail will appear in that folder when you
navigate to the folder in SimWrapper. Clicking on the thumbnail will
open that visualization full-screen.

Generally, a viz will require a specific set of inputs, and those inputs
are usually the result of some \emph{post-processing} of the raw
simulation outputs. It's up to you to do that post-processing and store
the files in the same folder as your config file.

Let's get started with some sample data.

\hypertarget{get-the-sample-data-and-open-it-in-simwrapper}{%
\subsection{1. Get the sample data and open it in
SimWrapper}\label{get-the-sample-data-and-open-it-in-simwrapper}}

\begin{itemize}
\tightlist
\item
  Download
  \href{https://github.com/simwrapper/simwrapper-example-project/archive/refs/heads/main.zip}{simwrapper-example-project.zip}
  from GitHub
\item
  Unzip the file somewhere you can find it easily - Desktop, home
  folder, etc.
\item
  Go to \href{https://simwrapper.github.io/site}{simwrapper.github.io}
  and click \texttt{Add\ folder...} and browse to the folder you just
  created. Grant access to the folder so the SimWrapper site can see the
  files!

  \begin{itemize}
  \tightlist
  \item
    (If you are using Firefox, \texttt{cd} to the data folder and run
    \texttt{simwrapper\ here} to start the local HTTP server)
  \end{itemize}
\end{itemize}

You should now see something similar to this:

Example data folder

\hypertarget{explore-the-samples}{%
\subsection{2. Explore the samples}\label{explore-the-samples}}

Each of the subfolders in the example project shows different map views
and capabilities of SimWrapper -- network link plots, statistical
charts, area maps (shapefiles), dashboards, and so on.

\begin{itemize}
\tightlist
\item
  Experiment with the various knobs and configuration settings to see
  how the visualizations can be manipulated
\item
  From your PC file browser, open up the \texttt{viz-*.yaml} files in
  each subfolder to see how each of the visualizations is defined in a
  readable text format.
\item
  Every visualization type has a different filename ``prefix'' to help
  you find them: e.g, \texttt{viz-map-*.yaml} are for shapefiles,
  \texttt{viz-link-*.yaml} are for MATSim network plots, and so on.
\item
  You can edit these YAML files, save, and click Reload on your browser
  to see how your changes affect the visualizations.
\end{itemize}

\hypertarget{create-a-dashboard-with-some-charts}{%
\subsection{3. Create a dashboard with some
charts}\label{create-a-dashboard-with-some-charts}}

The dashboards subfolder shows how you can combine multiple
visualizations into cohesive dashboards.

\begin{itemize}
\tightlist
\item
  The \texttt{dashboard-*.yaml} files define each individual tab in a
  dashboard. It's often nice to name them \texttt{dashboard-1-*.yaml},
  \texttt{dashboard-2-*.yaml} etc, to set them in the order that you
  like.
\item
  Dashboards are laid out in rows: Each row can have multiple panels.
  See the YAML files for how this works!
\end{itemize}

\hypertarget{configuring-dashboard-templates-for-multiple-run-folders}{%
\subsection{4. Configuring dashboard templates for multiple run
folders}\label{configuring-dashboard-templates-for-multiple-run-folders}}

In SimWrapper, everything is folder-based. So \texttt{viz-*.yaml} and
\texttt{dashboard-*.yaml} files in a folder will automatically be
detected and loaded based on their filenames.

If you want to define dashboards that will be used for \textbf{multiple
folders}, such as several runs for a particular project:

\begin{itemize}
\tightlist
\item
  \textbf{Create a folder} named \texttt{simwrapper} in the parent
  project directory.
\item
  Move all dashboard, viz, and template YAMLs into that folder
\item
  Tweak any file paths as necessary, so that relative file names resolve
  properly.
\item
  The base folder for a dashboard is the \emph{folder you are viewing},
  not the dashboard template folder.
\item
  You can have multiple \texttt{simwrapper} folders all the way up your
  folder hierarchy; dashboard panels will be generated based on
  filename, and each found dashboard will be displayed as a tab on the
  folder view.
\end{itemize}

\hypertarget{more-details-on-visualizations-and-their-yaml-files}{%
\subsection{5. More details on visualizations and their YAML
files}\label{more-details-on-visualizations-and-their-yaml-files}}

Here is an example YAML config file for a link-volume summary:

\textbf{viz-links-example.yaml:}

\begin{lstlisting}[]
  title: 'Taxi Passengers'
  description: 'Hourly passenger pickups'
  csvFile: 'vol_passengers.csv'
  geojsonFile: '../road-network.json.gz'
  projection: 'EPSG:25832'
  sampleRate: 0.10
\end{lstlisting}

This config names two files, a CSV of link volumes and a zipped JSON
file of the MATSim road network, and some parameters needed for the viz
to work. Those files are outputs of some post-processing scripts
described in the plugin docs.

If you wanted to look at several different link volume plots from the
same model run, (e.g.~for vehicle counts instead of passengers), you
would make a copy of this file, give it a different name, and edit the
\texttt{csvFile} parameter to point to the correct CSV.

This is a very different paradigm than most ``point and click'' GIS
tools, but we have found that the ability to script and cut/paste the
config files has been a huge time saver and also reduces manual errors.

\begin{quote}
Make sure that your files are allowed to be ``world-readable'' before
you publish anything to public-svn! Once files are pushed to public-svn,
they are not secured in any way; anyone on the internet can access them!
\end{quote}

\hypertarget{visualization-types}{%
\subsubsection{Visualization types}\label{visualization-types}}

Each visualization is described in the API Reference section of this
documentation, including how to post-process your outputs if necessary,
and how to define any configuration settings for your visualizations.
