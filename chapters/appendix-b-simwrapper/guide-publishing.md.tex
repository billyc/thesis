OK so you've curated your model runs, post-processed all your results,
and built your pretty dashboards! Now you want to show them to people.

We are experimenting with several different ways of publishing results
to the open web. Please send us feedback on which of these methods works
best for you, or if some other way of getting things online would be
better.

\hypertarget{speedrun-host-your-site-on-fly.io}{%
\subsection{Speedrun: Host your site on
fly.io}\label{speedrun-host-your-site-on-fly.io}}

There are many, many hosting options on the web but we have found that
https://fly.io is the most developer-friendly and has an extremely
generous free tier, so it's a great place to start. They seem to have
taken what Heroku used to be good at, and made it better.

Fly.io free tier currently (May 2022) includes:

\begin{itemize}
\tightlist
\item
  5 Gb storage
\item
  3 (or 10?) site URLs
\item
  100 GB data transfer per month free, then \$0.02/Gb over
\end{itemize}

Thus, you only need to watch the outbound data transfer to keep your
site free. This service doesn't have any redundancy or fault-tolerance,
so keep local copies of all your files in case their data center
explodes or catches on fire.

\hypertarget{first-time-setup}{%
\subsubsection{First time setup}\label{first-time-setup}}

\begin{enumerate}
\def\labelenumi{\arabic{enumi}.}
\item
  Install the \texttt{flyctl} utility with
  \href{https://fly.io/docs/getting-started/installing-flyctl/}{these
  instructions} but it is really this simple:

  \begin{itemize}
  \tightlist
  \item
    Mac: \texttt{brew\ install\ flyctl}
  \item
    Linux: \texttt{curl\ -L\ https://fly.io/install.sh\ \textbar{}\ sh}
  \item
    Windows:
    \texttt{iwr\ https://fly.io/install.ps1\ -useb\ \textbar{}\ iex}
  \end{itemize}
\item
  Then run \textbf{\texttt{flyctl\ auth\ signup}} to open a browser and
  set up your account. Yes, you need a credit card but it's free if you
  stay within the limits above.
\item
  No step 3!
\end{enumerate}

\hypertarget{push-your-site}{%
\subsubsection{Push your site}\label{push-your-site}}

Now that \texttt{flyctl} is set up on your machine you can launch your
first site!

\begin{enumerate}
\def\labelenumi{\arabic{enumi}.}
\tightlist
\item
  Clone the git repo at
  \texttt{https://github.com/simwrapper/simwrapper-example-project}
  which has the Dockerfile for SimWrapper ready to go
\item
  Copy your project folder -- everything you want published -- into the
  local \texttt{data} folder inside that repo
\item
  Run \texttt{flyctl\ launch} and answer the questions:

  \begin{itemize}
  \tightlist
  \item
    Give your site a useful name or it will give you something random
  \item
    Choose the datacenter; Frankfurt is a nice choice for Germany
  \item
    You do NOT need a database
  \item
    Yes, launch it now!
  \end{itemize}
\end{enumerate}

That's it. It will take a few minutes to upload all your files and
provision the site, and then\ldots{} that's it; it's live at
\textbf{your-site-name.fly.dev}

There are ways to give it your own domain name, etc. See the
\href{https://fly.io/docs}{fly.io/docs} for all the details.

\hypertarget{password-protect-your-site}{%
\subsubsection{Password-protect your
site}\label{password-protect-your-site}}

To password-protect a SimWrapper site using standard HTTP Basic
authentication, use the program \texttt{htpasswd} to generate a
username/password file named \texttt{.htpasswd} and copy that file into
the root of the built site using Docker. Github Pages does not support
this; you need to use a real web host such as Fly.io etc. that honors
basic authentication.

\texttt{htpasswd} is installed by default on Mac OS and is part of
\texttt{openssh} for other platforms.

\begin{enumerate}
\def\labelenumi{\arabic{enumi}.}
\tightlist
\item
  Use the program \texttt{htpasswd} to create username/password pairs
  you need:
\end{enumerate}

\begin{itemize}
\tightlist
\item
  \texttt{htpasswd\ -c\ .htpasswd\ {[}username{]}} and then enter the
  desired password. This will create/overwrite the file
  \texttt{.htpasswd}
\item
  If you need more than one user, drop the \texttt{-c} from the command
  and run again for each user:
  \texttt{htpasswd\ .htpasswd\ {[}anotheruser{]}}
\end{itemize}

\begin{enumerate}
\def\labelenumi{\arabic{enumi}.}
\setcounter{enumi}{1}
\tightlist
\item
  Add one line to the end of the Dockerfile, to copy the file into the
  root folder of the site:
\end{enumerate}

\begin{itemize}
\tightlist
\item
  \texttt{COPY\ .htpasswd\ \ /}
\end{itemize}

\begin{enumerate}
\def\labelenumi{\arabic{enumi}.}
\setcounter{enumi}{2}
\tightlist
\item
  Now run \texttt{flyctl\ launch} as above.
\end{enumerate}

\hypertarget{docker-based-sites}{%
\subsection{Docker-based sites}\label{docker-based-sites}}

SimWrapper has been packed up into a Docker image at
\texttt{simwrapper/site}, which is nothing more than the NGINX proxy
server with the latest SimWrapper code embedded, all carefully
configured to have the correct settings for serving the site and data
you provide.

On Docker Hub, we created the \texttt{simwrapper/site} docker image
which can be used anywhere that supports Docker. If you have Docker
installed on your local machine, or if you have a cloud provider that
lets you run Docker images, then you don't need anything other than:

\texttt{docker\ run\ -p\ 8080:8080\ -v\ \textasciitilde{}/my/data/folder:/data\ simwrapper/site}

That's it! Note that in the command above we mount a local volume to the
image at \texttt{/data} which is where SimWrapper expects your files to
be. For a cloud site, you would need to put that in accessible storage
somewhere. If you already know Docker, you already know how to do this.

\hypertarget{github-pages}{%
\subsection{GitHub Pages}\label{github-pages}}

Github is not a great place to store large files; it has a hard limit of
100Mb on file sizes, and generally doesn't work well for files over 20Mb
either. So, yes the SimWrapper website code and assets themselves can be
served from any static site provider at all, such as Github Pages. But
your model outputs might need to go somewhere else.

\hypertarget{instructions-for-using-github-pages-by-itself}{%
\subsubsection{Instructions for using GitHub Pages by
itself}\label{instructions-for-using-github-pages-by-itself}}

You can make your own clone of the SimWrapper website and host it
yourself on GitHub Pages, including all of your data files (if they are
each \textless{} 100Mb in size).

It will be hosted at yourusername.github.io/simwrapper, or you can
rename your clone to use a different endpoint slug other than
``/simwrapper''.

\begin{itemize}
\tightlist
\item
  Clone the SimWrapper Repo at
  \url{https://github.com/simwrapper/simwrapper.git}
\item
  Run \texttt{npm\ ci} to install all javascript dependencies
\item
  Copy all of your data files and configuration YAMLs into the
  \texttt{public/data} subfolder
\item
  You must edit two files, to tell GitHub what the endpoint of the URL
  is:

  \begin{itemize}
  \tightlist
  \item
    \texttt{vite.config.js}: change the \texttt{base} to be whatever
    your repo name is. For example to host things at
    ``username.github.io/simwrapper'', change this to
    \texttt{base:\ \textquotesingle{}/simwrapper/\textquotesingle{}}.
  \item
    \texttt{public/404.html}: change the \texttt{meta} refresh content
    value to show
    \texttt{"0;URL=\textquotesingle{}/simwrapper/\textquotesingle{}"}
  \end{itemize}
\item
  Run \texttt{npm\ run\ build}
\item
  Your site is now built in the local \texttt{dist} subfolder.
\item
  Push the content of that \texttt{dist} folder to a new
  \texttt{gh-pages} branch in your repo:
\end{itemize}

\begin{Shaded}
\begin{Highlighting}[]
\BuiltInTok{cd}\NormalTok{ dist}
\FunctionTok{git}\NormalTok{ init .}
\FunctionTok{git}\NormalTok{ add . }\KeywordTok{\&\&} \FunctionTok{git}\NormalTok{ commit }\AttributeTok{{-}m} \StringTok{"gh{-}pages"}
\FunctionTok{git}\NormalTok{ remote add origin git@github.com:username/reponame.git }\CommentTok{\# Put your username/reponame here}
\FunctionTok{git}\NormalTok{ push }\AttributeTok{{-}{-}force}\NormalTok{ origin master:gh{-}pages}
\end{Highlighting}
\end{Shaded}

Be sure to enable GitHub pages for your repo, and point it to the
\texttt{gh-pages} branch.

This worked as of fall 2021, if you have problems or things are broken
let us know!

\hypertarget{instructions-for-using-github-pages-your-own-file-server}{%
\subsubsection{Instructions for using GitHub Pages ( + Your own file
server)}\label{instructions-for-using-github-pages-your-own-file-server}}

\begin{itemize}
\item
  At VSP, we have a departmental Subversion server that is set up and
  maintained for us. The only settings we needed to tweak were to add
  ``CORS Headers'' which allow unfettered access to the site from other
  websites (e.g.~SimWrapper)
\item
  An NGINX proxy server is probably a more common choice if you don't
  already have Subversion lying around for some reason.
\item
  \textbf{For this to work,} you need to create your own build of
  SimWrapper and edit the \texttt{src/fileSystems.ts} file to point to
  your file server URL. Push that build to Github Pages.
\end{itemize}

\hypertarget{file-server-settings}{%
\subsection{File server settings}\label{file-server-settings}}

\hypertarget{cors-headers}{%
\subsubsection{CORS Headers}\label{cors-headers}}

These are the CORS headers needed for Apache or NGINX web servers. These
allow external websites to access the files on your server over HTTP.

\textbf{Your files will be visible on the Internet, so don't put things
there that should not be public.}

\begin{verbatim}
add_header Access-Control-Allow-Origin "*";
add_header Access-Control-Allow-Headers "Accept-Ranges,Range,*";
add_header Accept-Ranges "bytes";
\end{verbatim}

\hypertarget{nginx-settings}{%
\subsubsection{NGINX Settings}\label{nginx-settings}}

Here is an example configuration for NGINX, which includes the correct
CORS headers as well as the needed index.html redirection for
single-page-app URLs to work properly.

\begin{itemize}
\tightlist
\item
  The commented out section kills all caching; this was helpful for
  SFCTA which had their files on internal Windows servers.
\end{itemize}

\begin{verbatim}
server {
  # ...
  # you already have a server section
  # ...

  location / {
   try_files $uri $uri/ =404;
  }

  location /champ_runs/ {
    # Enable CORS and range requests for partial file downloads
    add_header Access-Control-Allow-Origin "*";
    add_header Access-Control-Allow-Headers "Accept-Ranges,Range,*";
    add_header Accept-Ranges "bytes";
    autoindex on;

    # optional: kill cache for the champ_runs windows share
    # add-header Last-Modified $date_gmt;
    # add-header Cache-Control 'private no-store, no-cache, must-revalidate,
    # proxy-revalidate, max-age=0';
    # add-header Pragma 'no-cache';
    # if_modified_since_off;
    # expires 0;
    # etag off;
  }

  location /simwrapper/ {
    # Redirect long URLs to the app index file which will handle them correctly
    try_files $uri $uri/ /simwrapper/index.html;
  }

# end server section
}
\end{verbatim}
