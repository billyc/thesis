This page details \textbf{build instructions} for setting up your own
instance of a SimWrapper website. This is for developers who want to
build and extend SimWrapper.

\begin{itemize}
\tightlist
\item
  If you are only interested in \emph{using SimWrapper} -- whether the
  official \href{https://vsp.berlin/simwrapper}{TU Berlin SimWrapper} or
  your own team's version of it -- then \emph{you do not need to read
  this page.}
\item
  Read \href{index.md}{this introduction} instead, or go straight to the
  \href{guide-dashboards.md}{dashboard guide}.
\end{itemize}

\hypertarget{foundational-technologies}{%
\subsection{Foundational technologies}\label{foundational-technologies}}

You will need to know this tech in order to hack on this website. If you
already know Python, learning TypeScript/Javascript is not that huge of
a leap, but it will definitely take some time to learn the full stack.

\begin{itemize}
\tightlist
\item
  \href{https://typescriptlang.org}{TypeScript} - typesafe Javascript.
  And if you don't know Javascript, you probably want to learn that
  first
\item
  \href{https://vuejs.org}{Vue} - the glue that connects UI elements to
  code. Similar to React but lightweight and awesome
\item
  \href{https://pugjs.org/}{Pug} - the template language used in our Vue
  files, far easier than writing bare HTML. You really only need Pug
  \href{https://pugjs.org/language/tags.html}{tags} and
  \href{https://pugjs.org/language/attributes.html}{attributes}
\item
  \href{https://deck.gl}{Deck.gl} - WebGL library for the fancy
  animations.
\end{itemize}

\hypertarget{setting-up-your-development-environment}{%
\subsection{Setting up your development
environment}\label{setting-up-your-development-environment}}

All tooling and source code is entirely free! You can build and hack on
SimWrapper without purchasing anything. The site uses
\href{https://nodejs.org}{Node.js} for build \& dependency management,
and was developed using \href{https://code.visualstudio.com/}{VS Code}.

\begin{enumerate}
\def\labelenumi{\arabic{enumi}.}
\tightlist
\item
  Install \href{https://nodejs.org/en/}{the latest ``LTS'' Version of
  Node.js}
\item
  Install \href{https://code.visualstudio.com/}{VS Code}
\item
  The following VS Code plugins are required. Find them in the VS Code
  plugin directory.

  \begin{itemize}
  \tightlist
  \item
    \textbf{Prettier} to force code-style consistency
  \item
    \textbf{Vetur}, the official Vue plugin.
  \end{itemize}
\end{enumerate}

\hypertarget{first-time-build}{%
\subsection{First time build}\label{first-time-build}}

Fork or clone the Github source repository, and then use the
\texttt{npm} node package manager to install all of the javascript
dependency libraries. The main SimWrapper source repo is at
\url{https://github.com/simwrapper/simwrapper} but \emph{you may want to
clone a different repo} if you are working on a different project such
as \href{https://github.com/ActivitySim/dashboard}{ActivitySim} or the
\href{https://github.com/sfcta/simwrapper}{SFCTA model}

\begin{enumerate}
\def\labelenumi{\arabic{enumi}.}
\tightlist
\item
  Clone the repo:
\end{enumerate}

\begin{itemize}
\tightlist
\item
  \texttt{git\ clone\ https://github.com/simwrapper/simwrapper\ \#\ or\ your\ fork}
\item
  \texttt{cd\ simwrapper}
\end{itemize}

\begin{enumerate}
\def\labelenumi{\arabic{enumi}.}
\setcounter{enumi}{1}
\tightlist
\item
  Install node dependencies
\end{enumerate}

\begin{itemize}
\tightlist
\item
  \texttt{npm\ ci}
\item
  This will spew lots of warnings but should ultimately finish without
  hard errors
\end{itemize}

Congrats! You now have SimWrapper installed and you are ready to start
hacking! 🎉✨

\hypertarget{development-commands}{%
\subsection{Development Commands}\label{development-commands}}

\textbf{npm run serve} -- Build and run a local copy of the site. This
is the main command you will use to start up the dev server, hack on
your changes in VS Code, and see the results. \texttt{npm\ run\ serve}
runs a local server with hot reload for testing, which usually listens
on http://localhost:8080

\textbf{npm run build} -- Compile and minify the build for production.
This command builds in ``production mode'', which performs some
optimizations, shakes out unused libraries, and catches some build-time
errors. You probably want to run this before you push anything live.

\textbf{npm run test:unit} -- Run unit tests. Ahahaha well\ldots{} I
have not written tests 👽 but the infrastructure is there to use
\texttt{jest}.

\hypertarget{hosting-on-github-and-other-places}{%
\subsection{Hosting on Github (and other
places)}\label{hosting-on-github-and-other-places}}

You've done all this work and you want your copy of SimWrapper to be
live on the internet? Yay that is exciting!

\textbf{Github Pages:} Here's how to use Github Pages to publish the
site.

Github pages allows you to publish static content (like SimWrapper) from
any repository you own. You place the built files in a special
\texttt{gh-pages} branch; the existence of that branch triggers Github
to start serving the files from your repository at the url
\texttt{my-github-userid.github.io/myrepo}. Thus if you were Github user
``billyc'' and you forked the repo into a repo called `simwrapper' then
it will be hosted at \{billyc\}.github.io/simwrapper

We wrote a deploy script that builds and pushes to Github Pages. Follow
these instructions to use it:

\begin{enumerate}
\def\labelenumi{\arabic{enumi}.}
\item
  First, edit the file \texttt{package.json} and modify the
  \texttt{deploy} line in the ``scripts'' section to make it match your
  userid/repository, e.g.~\texttt{billyc/simwrapper} -- DO NOT try to
  push to simwrapper/simwrapper thx :-)
\item
  Run \texttt{npm\ run\ build} to ensure that your site builds without
  any hard errors. The build goes into a folder named \texttt{dist}. Fix
  any errors and keep running this until it builds!
\item
  Run \texttt{npm\ run\ deploy} to rebuild the \texttt{dist} folder and,
  if successful, it will then force-push the contents to the
  \texttt{gh-pages} branch of your repository, overwriting any previous
  copy of the built site (don't worry the source code will remain safe
  on the master branch). In a couple moments your site will be live!
  Woot.
\end{enumerate}

\textbf{Non-Github Pages:} If you want to deploy somewhere other than
Github Pages, use \texttt{npm\ run\ build} to create the production
build in \texttt{dist} as above. You can push the entirety of that
folder to any static web host that you like, using their publishing
tools.

\hypertarget{project-layout}{%
\subsection{Project Layout}\label{project-layout}}

\begin{itemize}
\tightlist
\item
  \texttt{/scripts}: Housekeeping scripts go here. Most of these are
  used for postprocessing model results, written in Python, and also
  some build scripts are found here.
\item
  \texttt{/src}: all TypeScript and Vue files go here

  \begin{itemize}
  \tightlist
  \item
    \texttt{/src/assets}: images, .CSVs, etc that get packaged by the
    build process go here
  \item
    \texttt{/src/charts}: Dashboard chart types go here. Each chart type
    is a separate Vue component.
  \item
    \texttt{/src/components}: shared Vue components go here, there are
    lots of them
  \item
    \texttt{/src/js}: some typscript utility classes
  \item
    \texttt{/src/layers}: shared deck.gl layer files go here. These are
    generally written in JSX.
  \item
    \texttt{/src/plugins}: Each visualization plugin type gets its own
    folder here. To create your own plugin, copy one of these, rename
    its folder and main .vue file, and register it in
    \texttt{pluginRegistry.ts}. Read the plugin developer guide for
    details.
  \item
    \texttt{/src/views}: The Vue pages that render various site pages
    such as the home page. Vue pages are registered in
    \texttt{/src/router.ts}
  \end{itemize}
\item
  \texttt{/public}: These files are pushed as-is by the build process;
  they are not packaged in any way. They end up in the root folder `/'
  of the built site.
\end{itemize}

\hypertarget{thanks-for-your-interest}{%
\subsubsection{Thanks for your
interest!}\label{thanks-for-your-interest}}

Good luck and thanks for the help! --
\href{https://github.com/billyc}{Billy}
