\textbf{SimWrapper} is a unique, web-based data visualization tool for
researchers building disaggregate transportation simulations with
software such as \href{https://matsim.org}{MATSim} and
\href{https://activitysim.github.io}{ActivitySim}.

SimWrapper creates interactive dashboards and provides many statistical
views and chart types, just like other viz frameworks. But SimWrapper
also knows a lot about transportation, and has good defaults for
producing visualizations of network link volumes, agent movements
through time, aggregate area maps, scenario comparison, and a lot more.

You don't need to code in any language to use SimWrapper -- you point it
at your files, and write some small text (YAML) configuration files to
tell SimWrapper what to do. SimWrapper does the rest!

If you know JavaScript, the open-source code and plugin architecture of
SimWrapper allows you to fork the project and create your own
visualizations, too. But you don't need to know JavaScript if SimWrapper
already does what you need.

\hypertarget{how-simwrapper-works}{%
\subsection{How SimWrapper works}\label{how-simwrapper-works}}

SimWrapper is a web platform that can display either individual
full-page data visualizations, or collections of visualizations in
``dashboard'' format. It expects your simulation outputs to just be
regular files on your filesystem somewhere; there is no centralized
database or cloud server that you need to upload your results to.

To tell SimWrapper where your data files are:

\begin{itemize}
\tightlist
\item
  You can view files on your local computer directly in Google Chrome
  and Edge, or by running a tiny file server locally for Safari and
  Firefox (Chrome is recommended)
\item
  At VSP TU-Berlin, we have connected SimWrapper to our public file
  server, and use that for producing publicly accessible data dashboards
\item
  You can set up your own local SimWrapper server on your LAN or use
  internet file storage
\item
  If you have access to a remote compute cluster, you can see those
  files too if you mount the remote file system on your machine.
\end{itemize}

Once you point SimWrapper to your collection of files, some
visualizations will be immediately available --- depending on what
SimWrapper finds in your folder.

For other visualizations, you'll create tiny configuration files (in
YAML format) which tell SimWrapper what to load, how to lay out the
dashboards, and which provide all the config details to get it started.
These files can be collected in project folders and then will apply to
all runs in a set of folders, if you want.

\hypertarget{getting-started-with-simwrapper}{%
\subsection{Getting Started with
SimWrapper}\label{getting-started-with-simwrapper}}

\hypertarget{simwrapper-online-demo}{%
\subsubsection{1. SimWrapper online demo}\label{simwrapper-online-demo}}

Go to \url{https://vsp.berlin/simwrapper} and explore the project
dashboards on the home page there. You'll get an idea of what's possible
with SimWrapper.

\hypertarget{copying-the-demo-files-and-exploring-them-locally}{%
\subsubsection{2. Copying the demo files and exploring them
locally}\label{copying-the-demo-files-and-exploring-them-locally}}

We've created a SimWrapper example project with sample datasets and
configurations ready to go. - Download and unzip the latest
\href{https://github.com/simwrapper/simwrapper-example-project/archive/refs/heads/main.zip}{simwrapper-example-project.zip}
from the GitHub
\href{https://github.com/simwrapper/simwrapper-example-project}{example
project} - Open Google Chrome (for this demo, it's easiest to use Chrome
because it has an API for accessing local files on your computer;
Firefox and Safari don't) - Click \textbf{Add Local folder\ldots{}} and
navigate to the unzipped folder you just created. You'll see lots of
example visualizations and dashboards! - You can open any of datasets or
the \texttt{.yaml} configuration files in the archive to see how the
files and parameters are specified. Change things and see what happens.

\hypertarget{going-further}{%
\subsubsection{3. Going further}\label{going-further}}

The guides to the left cover the basics of getting up and running,
building your first visualizations and dashboards, and publishing
results online.

There is also an API Reference page for each type of visualization,
where you can find all of the configuration details for each type.

Finally, our
\href{https://github.com/simwrapper/simwrapper/issues}{GitHub Issue
Tracker} is the best place to ask questions and seek help.

\hypertarget{thank-you}{%
\subsection{Thank you!}\label{thank-you}}

It has been a lot of fun creating SimWrapper. This research was
developed at the Technische Universität Berlin Department of Transport
Telematics and Transport System Planning, under the direction of
Professor Kai Nagel.

Additional funding has been provided by the San Francisco County
Transportation Authority in San Francisco, California, and by member
agencies of the ActivitySim Consortium.

I'm glad you're here! Good luck with SimWrapper and thank you for the
feedback and contributions. -- \href{https://billyc.github.io/}{Billy}
