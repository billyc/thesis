\chapter*{Abstract}

For decades, transport modeling and transport simulation platforms have been coupled with data visualization tools to enable analysts and decisionmakers interpret the voluminous outputs that these simulations produce.

This dissertation documents the evaluation of web browser based data visualization techniques that integrate with open-source transport microsimulation tools such as MATSim and ActivitySim. Then it describes a new, complete, and novel open-source web platform that enables transport researchers to produce visualizations and data dashboards to support informed decisionmaking. It is organized into three parts:

\begin{itemize}
  \item Initial experiments using servers for processing and storage, and desktop web browsers for displaying simulation outputs
  \item A rethinking of the approach which eliminates the server-side software and moves almost all functionality into the web browser code, for a set of three project websites
  \item The synthesis of the above techniques into a cohesive and generically useful web-based platform
\end{itemize}

The first part describes several small tools exploring transport microsimulation datasets. Limits on dataset size and processing power inherent in desktop web browser-based solutions are considered. Strengths and weaknesses of the approach are considered in detail.

The second part describes a rethinking of this traditional client/server platform design toward an entirely browser-based system. Three project portals are described and address a wide range of topics from COVID-19 virus propagation to demand-responsive automated vehicle transport systems.

The technologies developed in the earlier parts are synthesized into a cohesive, general web-based platform that is described in the final part. The resulting tool, named ``SimWrapper'', can be used by analysts to assess and compare simulation runs, and also produces data dashboard portals that are suitable for a wide audience.

The dissertation closes with a discussion of the benefits and shortcomings of the web-based data visualization techniques developed as part of this research.
