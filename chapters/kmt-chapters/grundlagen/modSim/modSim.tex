% !TeX spellcheck = en_US
% !TEX root = ../../phd.tex

\todo[inline]{Modellierung und Simulation}
This chapter is dedicated to the modeling and simulation of (freight) traffic. This includes a general introduction to agent-based models (\cref{sec:gl-modsim-MATSim}). In the introduction of the software \acrshort{MATSim}, and some of its extensions which are used in this thesis are presented. 
This chapter concludes with a presentation of tour planning using the tour planning software \gls{jsprit} (\cref{sec:gl-modsim-vrp-jsprit}).


\section{Agent-based Modeling and Simulation}
\label{sec:gl-modsim-ABM}

\todo[inline]{abm schreiben}artinst
Agenten-basierte Modelle sind....

%%%%%%%%%%%%%%%%%%%%%%%%%%%%
%%%%%%%%%%%%%%%%%%%%%%%%%%%%

\section{The Multi-Agent Transport Simulation Software MATSim}
\label{sec:gl-modsim-MATSim}

\todo[inline]{MATSim-Teil selbst schreiben}


\subsection[MATSim]{MATSim}
\todo[inline]{anderen Namen finden für MATSim-main}
\rewrite{
The \acrfull{MATSim} is an open source software for agent-based transport simulation. It is an activity-based, extendable framework implemented in Java, which is designed for large-scale scenarios. In addition to a common base, several optional extensions 
are available (see \url{https://matsim.org/extensions}) \citep{MATSimBook}.
\gls{MATSim} is based on a co-evolutionary algorithm, where each iteration consists of the following three steps: Traffic simulation, scoring, and replanning. %(cf. Fig. \ref{fig:matsim}). 
\gls{MATSim} simulates agents -- normally persons -- and their daily plans, which consist of activities like home or work and trips between their activity locations. These plans are simulated and then scored based on their experienced performance. After the scoring some agents try to improve their plans, e.g. by selecting a different route or modifying activity start and end times. For this, several strategies are available to generate new or modified plans. After this replanning step, a new iteration of the simulation starts with the modified plans. After running the specified number of iterations, \gls{MATSim} terminates after the scoring step\citep{MATSimBook}
}


%%%%% TEXT von IK!!!!
%This \nameCref{simultaneous-sec:SimulationFramework} provides a brief overview of the activity-based transport simulation \gls{matsim}
%%\gls{matsim}%
%%%
%%\footnote{
%	%%
%	%Multi-Agent Transport Simulation, see www.matsim.org
%	%%
%	%}
%%
%which is used and extended in this thesis.
%\gls{matsim} is an open-source software developed under the terms of the GNU General Public License (\url{https://www.gnu.org/licenses/}), version 2 or later, published by the Free Software Foundation (\url{https://fsf.org}). The core code and several extensions are available on GitHub (\url{https://github.com/matsim-org/matsim}). The software structure follows a modular design which allows for an easy replacement or extension of the default functionality. The overview of \gls{matsim} provided in this \nameCref{simultaneous-sec:SimulationFramework} refers to the simulation setup which is most relevant for this thesis. The following is a combined and edited version of \gls{matsim} related descriptions in several articles that have been previously published, see, e.g., \citet{KaddouraNagel2016HeterogeneousVTTSPricing, KaddouraNagel2017CongestionPricing, KaddouraNagel2016CongestionNoiseOptimization, KaddouraKroegerNagel2015NoiseInternalization, Kaddoura2014CongestionPricingBerlin}, and \citet{NeumannEtAl2016MindTheGap, KaddouraEtAl2014AgentBasedPtOptimization} for the description related to \gls{pt}, or \citet{KaddouraEtAl2018SAVpricing} for the description related to \glspl{SAV}.
%
%In \gls{matsim}, each transport user is simulated as an individual agent. The agents' initial behavior has to be provided in the form of travel plans describing the daily activity patterns (e.g.,  home-work-shopping-home), the activity locations, the activity end times and information about the trips between these activities. The initial travel behavior is then modified applying an evolutionary iterative approach. In each iteration, (1) the travel plans are executed (Traffic simulation), (2) scored (Evaluation) and (3)~modified (Learning). (1) represents the agents' physical behavior; (2) and (3) describe the agents' mental behavior.
%
%\begin{enumerate}
%	\item \textbf{Traffic Simulation}
%	All travel plans are simultaneously executed and the transport users interact in the simulated physical environment. Traffic congestion and vehicle movements are simulated applying a queue model \citep{Gawron1998IterativeAlgorithmto}. Each road segment (link) is modeled as a \emph{First In First Out} queue with certain attributes, i.e, a free speed travel time $t_{free}$, a flow capacity $c_{flow}$ which limits the outflow \citep[in the literature often referred to as `bottleneck congestion', see, e.g.,][]{VanDenBerg2011Thesis}, and a storage capacity $c_{storage}$ which limits the number of vehicles on a link and causes spill-back. Each time step, typically 1~\gls{sec}, the state of each link's queue is updated. A vehicle is only moved from link $l_{1}$ to link $l_{2}$ if (i) the free speed travel time (given by the freespeed and length of link $l_{1}$) has passed, (ii) the inverse of the flow capacity has passed since the last vehicle left link $l_{1}$, and (iii) there is space on link $l_{2}$. Individual movements of agents can be aggregated to traffic flows that are found to be consistent with the fundamental diagram \citep[see, e.g.,][]{AgarwalEtcMixedTraffic}.
%	
%	\gls{matsim} also simulates the interaction of \textbf{\gls{pt}} users, transit vehicles and road traffic. The transit schedule provides all planned \gls{pt} operations. Transit vehicles may be delayed by (i) other road users and (ii) \gls{pt} users. Transit vehicles are included in the traffic flow simulation, thus, buses and cars may compete for the same limited road capacity. Each stop can be configured to either block the traffic or to allow overtaking whenever a bus stops. Depending on the vehicle type, i.e., the number of doors, and the assumption regarding simultaneous or sequential boarding/alighting, transit vehicles may be delayed by passengers at transit stops. In case a vehicle is fully loaded, additional boardings are denied and passengers will have to wait for the next transit vehicle.
%	Vehicles of one transit line can serve different tours. Consequently, the delay of one vehicle can be transferred to the following tour. In the case of a delay, the driver will try to follow the schedule by shortening dwell times (if no person wants to alight or board) as well as slack times.
%	For a detailed description of \gls{matsim}'s \gls{pt} dynamics, see \cite{Rieser2010} and \cite{Rieser2015PublicTransportMatsim}.
%	
%	The simulation of \textbf{\glspl{SAV}} is based on an existing module for dynamic vehicle routing problems \citep{Maciejewski2015DVRPInBook, MaciejewskiBischoffHoerlNagel2017DVRPTestbed} and an existing module for the simulation of taxis or \glspl{SAV} \citep{BischoffMaciejewski2016ANTAVBerlin}. \gls{SAV} users need to order an \gls{SAV} once they have left their activity location, wait for an available \gls{SAV} to arrive, get on the \gls{SAV} and get off the \gls{SAV} at the destination. \glspl{SAV} interact with each other and other vehicles, e.g., \glspl{CC}, and may get stuck in traffic.
%	
%	\item \textbf{Evaluation}
%	For each agent, the executed plan is evaluated based on agent-specific predefined behavioral parameters and utility functions. A plan's deterministic utility (often referred to as 'score') typically consists of two parts: (i) the generalized travel cost or trip-related disutility (e.g., travel time, monetary payments) and (ii) the utility gained from performing activities:
%	\begin{equation}
%		V_{p, i}= \sum_{a=1}^{A} \Big( 
%		%
%		V_{p,a}^{act} + V_{p,a}^{trip}
%		%
%		\Big) \ ,
%		\label{matsim-plan-utility}
%	\end{equation}
%	% 
%	where $V_{p,i}$ is the person $p$'s total utility (deterministic part) for a given plan $i$; $A$ is the total number of activities; $V_{p,a}^{act}$ is the
%	(positive) utility earned for performing activity $a$; and $V_{p,a}^{trip}$ is the (usually negative) utility earned for traveling to activity $a$. Activities are assumed to wrap around the 24-\acrshort{h}-period, that is,
%	the first and the last activity are stitched together. In consequence, there are as many trips between activities as there are activities. The first part typically considers the mode of transportation, the travel time, waiting, access and egress times and monetary costs (e.g., operation costs, fares, tolls).
%	The latter part is based on the approach by \cite{CharyparNagel2005ga4acts} where the marginal gain is typically positive but decreases with the duration spent at the activity location:
%	\begin{equation}
%		V_{p,a} = \beta^{perf}\cdot t^{typ}_{a} \cdot \ln\left( t^{perf}_{p,a} \middle/ t_{0,a} \right) \ ,
%		\label{matsim-eq:utilityPerf}
%	\end{equation}
%	% 
%	where $t^{perf}_{p,a}$ is the time person $p$ performs activity $a$, $t^{typ}_{a}$ is an activity's \emph{typical duration}, $\beta^{perf}$ is the marginal utility of performing an activity at its typical duration, and $t_{0,a}$ is defined as
%	\begin{equation}
%		t_{0,a} = t^{typ}_{a} \cdot \exp\left( -1 \right) \ ,
%		\label{decongestion-eq:t0}
%	\end{equation}
%	see \citet[][Sec.~97.4.2]{MATSimBook} for a discussion of this setting.
%	%a scale parameter 
%	%% \kai{Das stimmt vermutlich nicht mehr, oder?  (``uniform'' vs.\ ``relative'')}
%	%which is irrelevant as long as activities cannot be dropped.
%	%% \ihab{Auch bei 'relative' haben wir doch den $t_{0,a}$ parameter, und m.E. kann man den immer noch als 'scaling parameter' beschreiben. Ich hab jetzt einfach mal die Definition f\"ur $t_{0,a}$ hinzugef\"ugt. Das mit der priority spar ich mir hier mal, weil das an keiner Stelle relevant ist.}
%	Further activity-related constraints can easily be incorporated. An activity can only be performed during opening times. If an agent arrives at an activity location before or after the activity is open, the early or late arrival penalty results from the opportunity cost of time which is approximately equivalent to $\beta^{perf}$. Additionally, there may be a late arrival penalty $\beta^{late}$ which comes on top of the opportunity cost of time if an agent arrives after the \emph{latest arrival time} which can be specified for each activity.  Depending on the agent-specific travel plans, in particular the number and types of activities, agents are differently pressed for time, resulting in different \glspl{VTTS}.
%	%
%	
%	\item \textbf{Learning}
%	\label{introduction-sec:matsim-choice-dimensions}
%	For the majority of agents, the plan to be executed in the next iteration is chosen based on a multinomial logit model.
%	During the phase of choice set generation, some agents generate new travel plans by cloning an existing plan and changing parts of the cloned plan. The following provides an overview of the agents' basic choice dimensions which may be enabled separately or simultaneously.
%	\begin{itemize}
%		\item \textbf{ Route Choice:}
%		Based on the link-specific costs in the previous iterations such as travel times and toll payments, an agent's new transport route (sequence of links between one activity location and another one) is identified applying a least-cost path approach. To increase the diversity of identified transport routes, a random term may be added to the link-specific costs.
%		\item\textbf{Mode Choice:}
%		Agents randomly choose from the set of available modes (e.g., car, \gls{pt}, walk, bicycle, taxi) to replace the mode of an existing trip or (sub-)tour (e.g., the trip from home to work and the trip back home). Available modes may be specified for different agents or subpopulations. 
%		\item\textbf{Departure Time Choice:}
%		Activities' end times (departure times) are shifted by a random time period within a predefined maximum range. Typically, the predefined maximum time shift range is set to 2~\acrshort{h}.
%	\end{itemize}
%	The newly generated travel plan will then be executed in the next iteration.
%	Typically, the maximum number of travel plans per agent is limited, thus, the plan with the worst performance is discarded and plans that result in a higher utility are kept in the agent's choice set.
%\end{enumerate}
%%
%A repetition of these steps enables the agents to improve and obtain plausible travel plans, and the simulation results stabilize. Assuming that the set of each agent's travel plans represents a valid choice set, the system state is an approximation of the stochastic user equilibrium \citep{NagelFloetteroed2009IatbrResourceInBook}. Further details of the simulation framework \gls{matsim} are for example given in \cite{MATSimBook}.
%%%%% Ende TEXT von IK!!!!

%%%%%%%%%%%%%%%%%%%%%%%%%%%%

\subsection[MATSim:freight]{MATSim-Erweiterung: freight}
\label{gl-modsim-matsim:freight:subsec}
\todo[inline]{MATSim:Freight schreiben}

\rewrite{
Simulating freight traffic in agent-based traffic simulations is often done similarly to usual passenger traffic. The drivers of the vehicles are the individual agents with fixed plans of their tour. Typically, they can only optimize their own route between the various destinations. An adjustment of the sequence within the tour is rarely performed, and moving pickups or deliveries from one agent/vehicle to another is not possible \citep{ZilskeJoubert2015FreightTrafficInBook}.
 Because of this one needs to generate the tour plans in advance. The already mentioned toolkit \gls{jsprit} (see Sec \ref{sec:gl-modsim-vrp-jsprit}) is a \gls{vrp} solver that is integrated into \gls{MATSim}. That integration is done by using the \gls{MATSim} freight contrib \citep{ZilskeJoubert2015FreightTrafficInBook}.
}

%%%%%%%%%%%%%%%%%%%%%%%%%%%%

\subsection[MATSim:emissions]{MATSim-Erweiterung: emissions}
\label{gl-modsim-matsim-sec:emissions}
\todo[inline]{MATSim:emissions schreiben}

%%%%%%%%%%%%%%%%%%%%%%%%%%%%

\subsection[MATSim:roadpricing]{MATSim-Erweiterung: roadpricing}
\label{gl-modsim-matsim-sec:roadpricing}
\todo[inline]{MATSim:roadpricing schreiben}

%%%%%%%%%%%%%%%%%%%%%%%%%%%%

\subsection[MATSim:logistics]{MATSim-Erweiterung: logistics}
\label{gl-modsim-matsim-sec:logistics}
\todo[inline]{MATSim:logistics schreiben}
\kmt{Es ist noch keien eigenen contrib und auch unklar, ob es jemals eine eigene contrib wird...}

%%%%%%%%%%%%%%%%%%%%%%%%%%%%
%%%%%%%%%%%%%%%%%%%%%%%%%%%%

\section{Solving Verhicle Routing Problems}
\label{sec:gl-modsim-vrp}
\todo[inline]{jsprit schreiben}

\kmt{Übersetzen}
Für den Wirtschaftsverkehr ist die Tourenplanung von entscheidender Wichtigkeit. Mit ihrer Hilfe wird festgelegt, welche Kunden bzw. Kundenaufträge mit welchem Fahrzeug(typ) in welcher Tour und in welcher Reihenfolge innerhalb der Tour bedient werden. Nach einer Vorstelleung von verschiedenen Tourenplanugnsproblemen \todo{Crossref nutzen auch für Titel}  \todo{Abkürzung einführen} wird mit jsprit ein Tool zur Lösung von VRPs vorgestellt \todo{Crossref nutzen auch für Titel}

\subsection{Vehicle Routing Problem}
\label{sec:gl-modsim-vrp:VRP}
In the literature there are several different types of \glspl{vrp}. Some of them will be introduced in the following.

\rewrite{
\citet{IrnichEtAl2014VRP} define a \acrfull{vrp} as follows: Finding a plan to "determine a set of vehicle routes to perform [...] transportation requests with the given vehicle fleet at minimum costs" \citep{IrnichEtAl2014VRP}. A \gls{vrp} solves two interdependent main subproblems: (i) assigning requests to tours (clustering) and (ii) determining the optimal sequence in which the requests are served within a tour (routing) \citep{Prockl2010Informationsmanagement, Scheuerer2004HeuristikenTourenplanungPhd}.
The decision regarding optimal vehicle routing made by transport companies depends on various factors. Internal restrictions influencing the carrier can include the location of depot(s) and the available vehicle fleet at the depot(s). Each vehicle type has its own properties, e.g. capacity, fixed and variable costs and fuel consumption.
External restrictions and specifications offer a framework in which carrier have to operate. Customer requests (quantity, type of good, location and time window to bring goods) are very important specifications for generating tour plans \citep{JspritGithub}. Further constraints such as driving restrictions and tolls can exist due to political or court decisions \citep{Cerwenka2007Verkehrssystemplanung}.
The (expected) traffic situation has a direct impact on travel times. As a consequence, congestion can decrease service quality and can increase the costs of delivery for the carrier(s). Nevertheless, congestion is usually not considered in urban tour planning \citep{Ehmke2012RoutingInCityLogistics}.
For more information about different types for vehicle routing problems, see, e.g. \citep{TothVigo2014VRP}, \citep{Scheuerer2004HeuristikenTourenplanungPhd} or \citep{JspritGithub}.
}

\rewrite{ % Aus EFood-Paper
For transport companies, creating optimal vehicle routes depends on various factors. These factors can be internal and external. The internal factors include the location of depot(s) and the available fleet at the depot(s). 
Each vehicle type in a fleet has a set of given properties, e.g. carrying capacity and fixed and variable costs \cite{JspritGithub2018}. 
The most important external specifications are customer requests: quantity and type of goods, location and time window to deliver goods. 
Other external constraints are the road network and (expected) traffic situations, tolls and driving restrictions \cite{Cerwenka2007Verkehrssystemplanung}. Consideration of congestion is usually not part of urban tour planning \cite{Ehmke2012RoutingInCityLogistics}.
A VRP can be defined as follows: finding a plan to "determine a set of vehicle routes to perform [...] transportation requests with the given vehicle fleet at minimum costs"~\cite{IrnichEtAl2014VRP}.
Solving a VRP answers two main sub-problems for the carrier: (i) assigning requests to tours (clustering) and (ii) determining the optimal sequence in which the requests are served within a tour (routing)~\cite{Prockl2010Informationsmanagement, Scheuerer2004HeuristikenTourenplanungPhd}. 
More information regarding different types of vehicle routing problems can found e.g.\ in \cite{TothVigo2014VRP}, \cite{Scheuerer2004HeuristikenTourenplanungPhd} or \cite{JspritGithub2018}.
}

\kmtTodo{Weitere Unterarten der VRP vorstellen, Bilder aus ppt einfügen.}

\subsection{jsprit}
\label{sec:gl-modsim-vrp-jsprit}

\rewrite{
\Gls{jsprit} is an open source toolkit for solving \glspl{vrp}. For each \gls{vrp} there is a carrier which has a certain fleet and a list of requests. The fleet is located at the depot(s) from where they start and where they finish their tour. The fleet can be defined either by naming all available vehicles or by having an infinite fleet of defined vehicle type(s). When using the infinite fleet size, the composition of the fleet comes out as a result. The requests (or demand) can be defined in several ways: Either as \textit{service}s or as \textit{shipment}s. While services only have one location for delivering (or picking up) the goods - the other location is the depot - a shipment has two locations to define: from where to where the goods have to get transported. When applied to services, \gls{jsprit} can also solve \glspl{mdvrp}, which means that the algorithm also decides from which depot each request is served. \Gls{jsprit} iteratively optimizes the solution by ruining some parts of the solution and recreate a new one. 
The aim of \gls{jsprit} is, to reduce the costs for the carrier. Costs can be e.g. fixed costs for the vehicle including depreciation, insurance, ..., variable cost for distance and time a vehicle is on tour, or other costs such as tolls or penalties for missed time windows\citep{JspritGithub} .
}

\rewrite{ % Aus EFood-Paper
For solving VRPs, several toolkits are available. One of these is the open-source toolkit jsprit \cite{JspritGithub2018}. 
Jsprit optimizes the solution iteratively by ruining parts of the solution and recreating them. 
The objective function is to reduce the costs for the carriers. Inputs are fixed costs for the vehicle including depreciation, insurance, maintenance etc., variable cost for distance and travel time, and other costs such as tolls, or penalties for missed time windows~\cite{JspritGithub2018}.
Each VRP has a carrier with a certain fleet and list of requests. 
The fleet is located at the depot(s) and can be defined either with concrete vehicle numbers, or with an infinite fleet of specified vehicle types. 
When using the infinite fleet size, the fleet composition is also part of the solution.
The requests (demand) can be defined using one location (the other one is the depot) or explicitly by defining pickup and delivery location. 
Defining only one location opens the possibility to solve a multi-depot VRP (MDVRP), while defining both pickup and delivery locations allows the vehicle to reload goods during the tour.
Further information about both kinds of defining the requests can be found in \cite{MartinsTurnerNagel2019FreightMultipleToursABMTrans}.
}

%%%%%%%%%%%%%%%%%%%%%%%%%%%%
%%%%%%%%%%%%%%%%%%%%%%%%%%%%
