\chapter*{Zusammenfassung}

Seit Jahrzehnten werden Verkehrsmodellierungs- und Verkehrssimulationsplattformen mit Programmen zur Datenvisualisierung gekoppelt, um Analysten und Entscheidungsträger in die Lage zu versetzen, die umfangreichen Ergebnisse zu interpretieren, die diese Simulationen produzieren.

Diese Dissertation dokumentiert die Evaluierung von Webbrowser-basierten Datenvisualisierungstechniken, die mit Open-Source-Verkehrs-Mikrosimulationswerkzeugen wie MATSim und ActivitySim integriert werden können. Anschließend wird eine neue, vollständige und neuartige Open-Source-Webplattform beschrieben, die es Verkehrsforschern ermöglicht, Visualisierungen und Daten-Dashboards zur Unterstützung einer fundierten Entscheidungsfindung zu erstellen. Sie ist in drei Teile gegliedert:

\begin{itemize}
  \item Erste Experimente mit Servern für die Verarbeitung und Speicherung von Daten und mit Desktop-Webbrowsern für die Anzeige von Simulationsergebnissen
  \item Eine Überarbeitung des Ansatzes, bei der die serverseitige Software eliminiert und fast die gesamte Funktionalität in den Code des Webbrowsers verlagert wird, für eine Reihe von drei Projekt-Websites
  \item Die Synthese der oben genannten Techniken zu einer kohärenten und allgemein nützlichen webbasierten Plattform
\end{itemize}

Der erste Teil beschreibt mehrere kleine Tools zur Untersuchung von Verkehrssimulationsdatensätzen. Dabei werden die Grenzen der Datensatzgröße und der Rechenleistung, die Desktop-Webbrowser-basierten Lösungen innewohnen, berücksichtigt. Die Vorteile und Nachteile des Ansatzes werden im Detail betrachtet.

Der zweite Teil beschreibt ein Umdenken von dieser traditionellen Client/Server-Plattform hin zu einem vollständig browserbasierten System. Es werden drei Projektportale beschrieben, die ein breites Spektrum von Themen abdecken, von der Ausbreitung des COVID-19-Virus bis hin zu nachfrageabhängigen automatisierten Verkehrssysteme.

Die in den vorangegangenen Teilen entwickelten Technologien werden zu einer zusammenhängenden, allgemeinen webbasierten Plattform zusammengefasst, die im letzten Teil beschrieben wird. Das daraus resultierende Tool mit dem Namen “SimWrapper” kann von Analysten zur Bewertung und zum Vergleich von Simulationsläufen verwendet werden und erzeugt außerdem Daten-Dashboard-Portale, die für ein breites Publikum geeignet sind.

Die Dissertation schließt mit einer Diskussion der Stärken und Schwächender im Rahmen dieser Forschung entwickelten webbasierten Datenvisualisierungstechniken.


