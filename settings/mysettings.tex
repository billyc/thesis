%%%%%%%%%%%%%%%%%%%%%%%%%%%%%%%%%%%%%%
%% general
%\usepackage{cmbright} % serifenlose computer modern fonts
\usepackage[T1]{fontenc} % T1 fonts f\"ur gute pdf-Ausgabe
\setkomafont{disposition}{\normalcolor\bfseries} % font for headers same as normal text
\usekomafont{chapter}{\normalcolor\bfseries} % font for chapter headers same as normal text

%\usepackage[ansinew]{inputenc} % wegen deutschen Umlauten
\usepackage[utf8]{inputenc}

%\usepackage[automark]{scrpage2} % Koma Headers - deprecated....
\usepackage[automark]{scrlayer-scrpage} % Koma Headers - new version
\usepackage[english]{babel} % prepare for german AND english, last language loaded is default
\usepackage[autostyle=true, babel, english=american]{csquotes} % babel needs csquotes
%\usepackage{nag} % warn user on outdated packages
\usepackage[linktocpage,pagebackref=true]{hyperref}	% links in pdf, thumbnails
\usepackage{xspace}
\usepackage{diagbox}
\usepackage{soul} % emphasizing text, underlining
\usepackage{breakurl} % for broken urls in Bibliography when hyperref is in use
\usepackage[singlespacing]{setspace} % 1.5, change line spaces by \singlespacing \doublespacing
\RequirePackage{fix-cm} 	% error correction for standard fonts
\usepackage[Sonny]{fncychap}	% fancy chapters
\usepackage[titletoc]{appendix}

\usepackage[font={footnotesize}]{caption}

%%%%%%%%%%%%%%%%%%%%%%%%%%%%%%%%%%%%%%
%% tables
\usepackage{multicol} % multiple columns in tables
\usepackage{multirow} % multiple rows in tables
\usepackage[margin=10pt,labelfont=bf]{caption} % table headers
\usepackage{hhline}	% horizontal lines
\usepackage{longtable} % pagebreak tables
\usepackage{booktabs} % bold table lines, e.g. \toprule
\usepackage{tabularx} % neue Tabular-Umgebung
\usepackage{etoolbox} % adjust font size for tables
\BeforeBeginEnvironment{tabular}{\begin{center}\footnotesize} % adjust font size for tables
\AfterEndEnvironment{tabular}{\end{center}} % adjust font size for tables
%--------------------------------------------------------------------
% table settings
%
\newcommand{\colortableformat}{
	\renewcommand{\arraystretch}{1.3}
	\small
	\sffamily
	\rowcolors{1}{gray!15}{gray!25}% {startrow}{oddrowcolor}{evenrowcolor}
}

%%%%%%%%%%%%%%%%%%%%%%%%%%%%%%%%%%%%%%
%% math, symbols
\usepackage{amsmath} % AMS Math like brackets, integrals...
\usepackage{amssymb} % AMS-Symbols v2.0
\usepackage{fixmath} % big greek letters italic in math mode
\usepackage{array} % matrices
\usepackage{units} % includes nicefrac, nicer fracs for one line, SI-Units
\usepackage{trfsigns} % symbole fÌr transformationen
\usepackage{textcomp} % einfache Sonderzeichen, z.B. \texteuro
\usepackage{gensymb} % correct greek letters in units,\micro instead of \mu
\usepackage[integrals]{wasysym} % for integrals like \oiint
\usepackage[version=3]{mhchem} % easy typesetting of chemical formulae
\usepackage{eurosym} %Eurosymbol

%%%%%%%%%%%%%%%%%%%%%%%%%%%%%%%%%%%%%%
%% graphics
%\usepackage[activate]{pdfcprot} % use margin kerning (character protruding) (Opt. Randausgleich)
\usepackage{microtype} % character protruding, font expansion - instead of pdfcprot
\usepackage{graphicx} % include graphics
\usepackage{wrapfig} % graphics in text
%\usepackage{floatflt} % graphics/tables in text
%\usepackage{rotating} % rotating elements
\usepackage{listings} % for programming source code
\usepackage[svgnames]{xcolor} % colors for listings
\usepackage{psfrag}	% Text in .eps Grafiken ersetzen
\usepackage{subfig}
%\usepackage{subcaption} % can't be used together with subfig
\usepackage{paralist} % in line bullets/points
\usepackage[shortlabels]{enumitem} % different labels for enumerator
\usepackage{bookmark}
\usepackage{tikz}

% clever references, has to be loaded last
\usepackage[capitalise]{cleveref}
\crefname{part}{Part}{Parts}
\Crefname{part}{part}{parts} % only used for ``In this part, ...'' (\nameCref{})
\crefname{chapter}{Ch.}{Chs.}
\Crefname{chapter}{chapter}{chapters} % only used for ``In this chapter, ...'' (\nameCref{})
\crefname{section}{Sec.}{Secs.}
\Crefname{section}{section}{sections} % only used for ``In this section, ...'' (\nameCref{})
\crefname{table}{Tab.}{Tabs.}
\crefname{figure}{Fig.}{Figs.}
\crefname{appendix}{App.}{App.}
\crefname{equation}{Eq.}{Eqs.}

\bookmarksetup{startatroot}
\addtocontents{toc}{\bigskip}% adds a little space in the printed table of contents to visually separate the final chapter from the preceding "part"

% AbstÀnde fÌr Seitenzahlen im Inhaltsverzeichnis erhöhen
\makeatletter
\renewcommand*{\@pnumwidth}{2em} % Breite der Box fÌr Seitenzahlen im Inhaltsverzeichnis erhöhen
\makeatother
%**********************************************

%%%%%%%%%%%%%%%%%%%%%%%%%%%%%%%%%%%%%%
%% bibliography
\usepackage{natbib}
%\usepackage[backend=bibtex,
%			natbib=true,
%			style=authoryear-comp,
%			maxcitenames=2,
%			maxbibnames=99,
%			mincitenames=1,
%			uniquelist=minyear,
%			hyperref=true,
%			arxiv=abs,
%			firstinits=true,
%			mincrossrefs=99,
%			sortcites=false,
%			backref=true]{biblatex}

%%%%%%%%%%%%%%%%%%%%%%%%%%%%%%%%%%%%%%
%% layout
\usepackage[top=4cm,left=3.5cm,right=2.5cm,bottom=4cm]{geometry}

%%%%%%%%%%%%%%%%%%%%%%%%%%%%%%%%%%%%%%
%%%%%%%%%%%%%%%%%%%%%%%%%%%%%%%%%%%%%%
%%%%%%%%%%%%%%%%%%%%%%%%%%%%%%%%%%%%%%

\definecolor{matsimblue}{RGB/cmyk}{50,100,175/0.714,0.429,0.000,0,314}

%% to-do notes
\usepackage[]{todonotes}
%\usepackage[disable]{todonotes}


\def\kmt#1{\textcolor{blue}{[[#1 -- kmt]]}}
\def\kai#1{\textcolor{darkgreen}{[[#1 -- kai]]}}

\def\rewrite#1{\textcolor{teal}{[[#1 -- Umschreiben - ist 1:1 aus Paper]]}}

\newcommand{\kmtTodo}[2][]
{\todo[inline,caption={#2}, size=\scriptsize, color=blue!30, #1]{\renewcommand{\baselinestretch}{0.5}\selectfont#2 KMT\par}}

\newcommand{\kaiTodo}[2][]
{\todo[inline,caption={#2}, size=\scriptsize, color=darkgreen!30, #1]{\renewcommand{\baselinestretch}{0.5}\selectfont#2 KN\par}}

%% mnotes
%\def\mnote#1{\medskip\centerline{\hfill\textcolor{red}{\small #1}}}
\def\mnote#1{\relax}

\usepackage[draft,markup=underlined
,addedmarkup=colored
%%,deletedmarkup=xout
%%,commandnameprefix=ifneeded
]{changes}
\let\delete=\deleted
\let\remove=\deleted
\let\removed=\deleted
\let\add=\added
\let\replace=\replaced
\let\change=\replaced
\let\changed=\change
\makeatletter
\@namedef{Changes@AuthorColor}{magenta}
\colorlet{Changes@Color}{magenta}
\makeatother

%%%%%%%%%%%%%%%%%%%%%%%%%%%%%%%%%%%%%%
%%%%%%%%%%%%%%%%%%%%%%%%%%%%%%%%%%%%%%
%%%%%%%%%%%%%%%%%%%%%%%%%%%%%%%%%%%%%%

\graphicspath{{./title/graphics/}{./introduction/graphics/}{./congestion/ptDelayPricing/}{./congestion/congestionPricing/}{./congestion/vtts/}{./noisePricing/noise/}{./noisePricing/ACP/graphics/}{./noisePricing/MCP/}{./severalExternalEffects/simultaneous/}{./congestion/decongestion/}{./severalExternalEffects/cna/}{./avOpt/}}

%%%%%%%%%%%%%%%%%%%%%%%%%%%%%%%%%%%%%%
%%%%%%%%%%%%%%%%%%%%%%%%%%%%%%%%%%%%%%
%%%%%%%%%%%%%%%%%%%%%%%%%%%%%%%%%%%%%%

\makeatother

%--------------------------------------------------------------------
% glossary
\usepackage[toc, acronym]{glossaries}
\makeglossaries
% !TEX root = ../phd.tex

%%%%%%%%%%%%%%%%%%%%%%%%%%%%%%%%%%%%%%%%%%%%%%%%%%%%%%
%%%%%%%%%%%%%%%%%%%%%%%%%%%%%%%%%%%%%%%%%%%%%%%%%%%%%%
% acronyms
%%%%%%%%%%%%%%%%%%%%%%%%%%%%%%%%%%%%%%%%%%%%%%%%%%%%%%
%%%%%%%%%%%%%%%%%%%%%%%%%%%%%%%%%%%%%%%%%%%%%%%%%%%%%%

%\newacronym[description={Extensible Markup Language, see
	%	\url{www.w3.org/XML}}]{xml}{XML}{Extensible Markup Language}

%Hinweis: die Description  ist nur notwendig, wenn diese von der Langfassung (letztes {}-Paar) abweicht.
%Hinweis:  Pluraldefinitonen sind nur notwendig, wenn diese nicht durch dranhängen eines "s" gebildet werden..

\newacronym{API}{API}{Application Programming Interface}

\newacronym{BEV}{BEV}{Battery Electric Vehicle}

\newacronym{DSL}{DSL}{Domain-specific language}

\newacronym{DRT}{DRT}{Demand responsive transport}

\newacronym{eu}{EU}{European Union}

% \newacronym{FCEV}{FCEV}{\acrlong{FC}  Engine Vehicle}

\newacronym[description={GeoServer is an open source server for sharing geospatial data, see \url{https://geoserver.org}}]
{GeoServer}{GeoServer}{GeoServer geospatial data server}

\newacronym{hbefa}{HBEFA}{Handbook on Emission Factors for Road Transport}

\newacronym{HTML 5}{HTML 5}{Hypertext Markup Language, version 5}

\newacronym[description={Hypertext Transfer Protocol. HTTP and its encrypted version HTTPS are the primary data transfer protocols used by websites}]
{HTTP}{HTTP}{Hypertext Transfer Protocol}

\newacronym{ICEV}{ICEV}{Internal Combustion Engine Vehicle}

\newacronym[description={Java programming language, see \url{www.java.com}}]
{java}{Java}{Java programming language}

\newacronym{KPI}{KPI}{Key Performance Indicator}

\newacronym{lez}{LEZ}{Low Emissions Zone}

\newacronym[description={JavaScript programming language, see \url{www.javascript.info}}]
{JavaScript}{JavaScript}{JavaScript programming language}

\newacronym[description={Multi-Agent Transport Simulation, see \url{www.matsim.org}}, sort=matsim]
{MATSim}{\mbox{MATSim}}{Multi-Agent Transport Simulation}

% \newacronym[description={jsprit, an open-source \acrlong{vrp} - solver, see \url{http://jsprit.github.io/}}]
% {jsprit}{jsprit}{jsprit}

\newacronym{PAVE}{PAVE}{PAVE Project: Potential of Automated Transport Systems}

\newacronym{SPA}{SPA}{Single Page Application}

\newacronym{UI}{UI}{User Interface}

\newacronym{UX}{UX}{User Experience}

\newacronym[description={VSP is Verkehrssystemplanung und Verkehrstelematik, the transport planning and transport telematics chair at Technische Universität Berlin}]
{VSP}{VSP}{Verkehrssystemplanung und Verkehrstelematik, Technische Universität Berlin}

\newacronym[description={Extensible Markup Language, see \url{www.w3.org/XML}}]
{xml}{XML}{Extensible Markup Language}

\newacronym[description={Yet Another Markup Language, see \url{yaml.org/}}]
{YAML}{YAML}{Yet Another Markup Language}

%%%%%%%%%%%%%%%%%%%%%%%%%%%%%%%%%%%%%%%%%%%%%%%%%%%%%%
%%%%%%%%%%%%%%%%%%%%%%%%%%%%%%%%%%%%%%%%%%%%%%%%%%%%%%
% symbols/units
%%%%%%%%%%%%%%%%%%%%%%%%%%%%%%%%%%%%%%%%%%%%%%%%%%%%%%
%%%%%%%%%%%%%%%%%%%%%%%%%%%%%%%%%%%%%%%%%%%%%%%%%%%%%%

\newacronym[description={particulate matter}, %sort={pm},
nonumberlist]{pm}{\ensuremath{\mathit{PM}}}{particular matter}

\newacronym[description={nitrogen oxides}, %sort={nox},
nonumberlist]{nox}{\ensuremath{\mathit{NO_x}}}{nitrogen oxides}

\newacronym[description={sulfur dioxide},
nonumberlist]{so2}{\ensuremath{\mathit{SO_2}}}{sulfur dioxide}

\newacronym[description={non-methane hydrocarbons},
nonumberlist]{nmhc}{\ensuremath{\mathit{NMHC}}}{non-methane hydrocarbons}

\newacronym[description={noise day-evening-night index}, %sort={Lden},
nonumberlist]{Lden}{\ensuremath{\mathit{L_{den}}}}{day-evening-night noise index, see Environmental Noise Directive of the European Union \cite{EU2002END}}

\newacronym[description={carbon dioxide}, %sort={co2},
nonumberlist]{co2}{\ensuremath{\mathit{CO_2}}}{carbon dioxide}

\newacronym[description={kilometer, 1000~\acrshort{m}},
nonumberlist]{km}{\ensuremath{km}}{kilometer}

\newacronym[description={square kilometer, $\acrshort{km}^{2}$},
nonumberlist]{sqkm}{\ensuremath{sqkm}}{square kilometer}

\newacronym[description={meter, the SI base unit of length},
nonumberlist]{m}{\ensuremath{m}}{meter}

\newacronym[description={million, $1 \times 10^6$ },
nonumberlist]{mil}{\ensuremath{M}}{million}

\newacronym[description={thousand, $1 \times 10^3$ },
nonumberlist]{k}{\ensuremath{k}}{thousand}

\newacronym[description={Vehicle Kilometer},
nonumberlist]{vkm}{\ensuremath{vkm}}{Vehicle Kilometer}

%\newacronym[description={Vehicle Hour},
%nonumberlist]{vh}{\ensuremath{vh}}{Vehicle Hour}

\newacronym[description={Ton, 1000~\acrshort{kg}},
nonumberlist]{ton}{\ensuremath{t}}{Ton}

\newacronym[description={kilogram, the SI base unit of mass},
nonumberlist]{kg}{\ensuremath{kg}}{kilogram}
%
\newacronym[description={gram, 1/1000~\acrshort{kg}},
nonumberlist]{g}{\ensuremath{g}}{gram}

\newacronym[description={hour, 60~\acrshort{min}},
nonumberlist]{h}{\ensuremath{h}}{hour}

\newacronym[description={minute, 60~\acrshort{sec}},
nonumberlist]{min}{\ensuremath{min}}{minute}

\newacronym[description={second, the SI base unit for time},
nonumberlist]{sec}{\ensuremath{sec}}{second}

\newacronym[description={utility unit},
nonumberlist]{util}{\ensuremath{util}}{util}

\newacronym[description={Euro},
nonumberlist]{EUR}{\ensuremath{\mathit{EUR}}}{Euro}

\newacronym[description={passenger car unit},
nonumberlist]{pcu}{\ensuremath{\mathit{pcu}}}{passenger car unit}

%--------------------------------------------------------------------
%listings (code snippets)
%\usepackage{listings}
%\renewcommand{\lstlistlistingname}{List of Listings}
%\lstset{numbers=left, numberstyle=\footnotesize, numbersep=5pt, basicstyle=\footnotesize\sffamily}
%\lstset{language=Java}
%--------------------------------------------------------------------

%%%%%%%%%%%%%%%%%%%%%%%%%%%%%%%%%%%%%%
%%%%%%%%%%%%%%%%%%%%%%%%%%%%%%%%%%%%%%
%%%%%%%%%%%%%%%%%%%%%%%%%%%%%%%%%%%%%%

% author, title, date of thesis
\newcommand*{\Title}{YY Web-Based Data Visualization in Support of Agent Based Microsimulation Models}

\newcommand*{\Autor}{William A. Charlton}
\newcommand*{\Datum}{Berlin 2023}
\title{\Title}
\author{\Autor}
\date{\Datum}

\newcommand{\tfk}[1]{\textsl{\texttt{#1}}}
\newcommand{\fett}[1]{\textbf{#1}}
\newcommand{\kursiv}[1]{\textit{#1}}
\newcommand{\pbb}{\parbox}
\newcommand{\sst}{\scriptstyle}

% \def\tightlist{}

\providecommand{\tightlist}{%
  \setlength{\itemsep}{0pt}\setlength{\parskip}{0pt}}

\def\umbruch{\clearpage}
\definecolor{darkgreen}{RGB}{0 153 0}

% fÃŒr Standardumgebungen
%\renewcommand{\labelitemi}{*} % AufzÀhlungszeichen definieren

%%%%%%%%%%%%%%%%%%%%%%%%%%%%%%%%%%%%%%%
%%%%%%%%%%%%%%%%%%%%%%%%%%%%%%%%%%%%%%%
%%%%%%%%%%%%%%%%%%%%%%%%%%%%%%%%%%%%%%%
%%%%%%%%%%%%%%%%%%%%%%%%%%%%%%%%%%%%%%%

% Allgemeine Schalter - Änderung von Standardeinstellungen
\frenchspacing								% keine lÀngeren Leerzeichen nach Satzende/AbkÌrzungen mit Punkt
%\setlength{\parindent}{0pt}						% kein Einzug bei neuem Absatz
\setlength{\parindent}{1.5ex}						% kein Einzug bei neuem Absatz
%\setlength{\parskip}{1.5ex plus0.5ex minus 0.5ex}	% Abstand zwischen 2 AbsÀtzen
\setlength{\parskip}{0.25ex plus0.25ex minus 0.25ex}	% Abstand zwischen 2 AbsÀtzen


% verwende das paket setspace statt baselinestretch, Vorteil: AbstÀnde in Fußzeilen und
% listenumgebungen etc. bleiben erhalten
%\renewcommand{\baselinestretch}{1.2}			% Zeilenabstand

% WortabstÀnde optimal einstellen (use instead of \sloppy) - Verhindern von rausragenden Zeilen
\tolerance 1414
\hbadness 1414
\emergencystretch 1.5em %1.5em
\hfuzz 0.3pt
\widowpenalty=10000
\vfuzz \hfuzz
\raggedbottom
\brokenpenalty=10000						% Trennung bei Seitenumbruch

\setlength{\headheight}{1cm} 					% Höhe der Kopfzeile
\addtolength{\footnotesep}{2pt}					% Abstand der Fußnote zur Trennlinie

% Setze Überschriftentiefe
\setcounter{secnumdepth}{3}					% Nummerierung der Kapitel
\setcounter{figure}{4}							% Bilder
\setcounter{tocdepth}{1}						% Gliederungsebene fÃŒr Inhaltsverzeichnis

% Einstellungen fÃŒr Kopf- und Fußzeilen
\pagestyle{scrheadings}       					% nutze scrheader
\clearscrheadings             						% lösche alle vorhandenen header
%\addtokomafont{pagehead}{\normalsize\upshape}  % nichtkursive Kopf-/Fußzeilen
\setheadsepline{.1pt}							% trennlinie oben
\setfootsepline{.1pt}							% trennlinie unten

\automark[section]{chapter}   					% links chapter, rechts section

% variere nach ein-/zweiseitig
\makeatletter
\if@twoside												% bei zweiseitig
	\lehead{\leftmark}             % setze Kapitel linke Seite oben
	\rohead{\rightmark}           % setze Abschnitt rechte Seite oben
	\lefoot{\pagemark}            % Seitennummer unten links
	\rofoot{\pagemark}            % Seitennummer unten rechts
	%\lofoot{\Autor}       		% Name des Verfassers nur linke Seite
\else														% einseitig
	\ihead{\leftmark}            	% setze linke kopfzeile
	\ohead{\rightmark}           	% setze rechte kopfzeile
	\ofoot{\pagemark}             % seitennummer unten rechts
	%\ifoot{\Autor}       		% Name des Verfassers unten links
\fi

% Bei Kapiteln ohne Subsection wird Kapitelname eingeblendet, nutze
% \markright{}, um den \rohead freizulassen

\hypersetup{
			pdftitle={\Title},
			pdfsubject={Dissertation, TU Berlin},
			pdfauthor={William Charlton},
			pdfkeywords={data visualization; transport; agent-based transport simulation},
			pdfdisplaydoctitle = {true},
			pdffitwindow = {true},
			pdfpagelayout = {TwoPageRight},
			pdflang={en},
			colorlinks=true,
			citecolor=matsimblue,
			linkcolor=matsimblue,
			urlcolor=matsimblue,
			bookmarksnumbered=true,
bookmarksdepth=4}

\makeatother

\newcommand{\perf}{\it perf}
